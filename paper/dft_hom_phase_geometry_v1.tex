\documentclass[11pt,a4paper]{article}

% --- Packages ---
\usepackage[margin=1in]{geometry}
\usepackage{amsmath,amssymb,amsfonts}
\usepackage{graphicx}
\usepackage{tabularx}
\usepackage{ragged2e}
\newcolumntype{Y}{>{\RaggedRight\arraybackslash}X}
\usepackage{booktabs}
\usepackage{bm}
\usepackage{enumitem}
\usepackage{physics}
\usepackage{siunitx}
\AtBeginDocument{\let\qty\SI}
\usepackage{hyperref}
\usepackage[numbers,sort&compress]{natbib}
\usepackage{tikz}
\usetikzlibrary{arrows.meta}


% --- Meta info ---
\title{Two-Photon Bunching from Geometric Phase Diffusion:\\
Hong--Ou--Mandel Interference without Wavefunctions}

\author{A.R. Wells}

\date{December 2025}

% --- Convenience macros ---
\newcommand{\Var}{\operatorname{Var}}
\newcommand{\E}{\mathbb{E}}
\newcommand{\cN}{\mathcal{N}}
\providecommand{\dd}{\mathrm{d}}

\begin{document}
\setcounter{figure}{0}
\renewcommand{\thefigure}{\arabic{figure}}
\maketitle

\begin{abstract}
We present a constructive trajectory-based account of two-photon
interference in the Hong--Ou--Mandel (HOM) experiment within
Dual-Frame Theory (DFT), a geometric framework in which systems follow discrete paths carrying an internal phase coordinate. Quantum interference phenomena such as the Hong--Ou--Mandel (HOM) effect are
usually treated as direct manifestations of the standard Hilbert-space
formalism: bosonic symmetrization, wavefunction superposition, and measurement
operators on tensor-product spaces. In this work we show that HOM bunching can
instead be derived from a discrete trajectory model with a geometric phase
degree of freedom. Within Dual-Frame Theory (DFT), each system follows a
definite trajectory carrying a T-frame phase coordinate whose evolution obeys a
simple motion budget, $(\Delta x)^2 + (\Delta \theta)^2 = 1$. We demonstrate
that, for correlated pairs, this microscopic rule generates intrinsic phase
diffusion with coefficient $D \approx 1$ and an additional $N^2$ contribution
set by source bandwidth. The resulting coherence envelope
$c(N) = \exp(-a N - b N^2)$, with $a = D/2$ and $b = \sigma_\omega^2/2$,
directly yields a Hong--Ou--Mandel coincidence dip in quantitative agreement
with the standard quantum prediction, without invoking wavefunctions or
superposition postulates. At the level treated here, the resulting predictions for HOM interference are quantitatively indistinguishable from those of standard quantum mechanics.
Accordingly, the present work should be understood as a constructive reformulation of the HOM mechanism within a geometric trajectory-based framework, rather than a demonstration of new physics. Potential deviations are tied to the continuum limit of the discrete motion budget and are discussed as targets for future experimental tests. This separates universal geometric dynamics from experiment-dependent distinguishability, and shows that multi-particle quantum
coherence can emerge from phase geometry on discrete trajectories. We conclude by outlining consequences for Bell-type tests and for the continuum limit of DFT.
\end{abstract}

%--------------------------------------------------
\section{Introduction}
\label{sec:intro}

Quantum interference phenomena such as the Hong--Ou--Mandel (HOM) effect are
typically regarded as hallmarks of the standard quantum formalism: bosonic
symmetrization, wavefunction superposition, and measurement operators on
Hilbert-space tensor products. 
~\cite{Hong1987,MandelWolf1995,GerryKnight2005} 
Yet the mathematical structure of these
interference rules is not itself explanatory. It encodes how amplitudes
combine, not why they should combine with that particular structure or how the
underlying correlations arise in the first place.

This raises a foundational question:
\emph{Is HOM interference a primitive axiom of the Hilbert-space calculus, or
can it be derived from a deeper constructive model?}

A constructive model would
\begin{itemize}
  \item assign definite microscopic structure to quantum systems,
  \item produce interference as an emergent collective phenomenon,
  \item reproduce quantitative predictions of standard quantum mechanics (QM),
  \item identify which aspects of coherence are universal and which are
        experiment-dependent, and
  \item suggest regimes where its predictions might deviate from those of QM.
\end{itemize}
The goal of this work is to demonstrate that such a constructive account
exists.

The present work sits alongside other trajectory-based and
geometric approaches to quantum phenomena, such as Bohmian mechanics,
stochastic mechanics, and various hidden-variable or pilot-wave
frameworks.~\cite{Holland1993,DurrTeufel2009,Nelson1966,Nelson1967}
Our goal here is not to survey that broad landscape but to
demonstrate that, within a specific geometric ontology (DFT), a
standard benchmark of two-photon interference can be reproduced with
explicit, testable structure at the level of microscopic trajectories.
A detailed comparison with other interpretational frameworks is left
for future work.

Compared with Bohmian mechanics, which augments the wavefunction with
definite particle positions guided by a pilot wave, DFT dispenses with
the wavefunction entirely and treats phase geometry on trajectory
space as fundamental.~\cite{Holland1993,DurrTeufel2009} Stochastic mechanics, in turn, models quantum
behavior via diffusion processes in configuration space driven by
classical-like noise;~\cite{Nelson1966,Nelson1967} DFT differs in that its ``diffusion'' arises
from a compact internal phase coordinate constrained by a geometric
motion budget rather than from external noise on spatial coordinates.
Thus DFT is closer in spirit to a geometric reformulation of phase
structure than to a conventional hidden-variable theory on
configuration space.

The HOM calculation in this paper should therefore be viewed as a
proof-of-concept that the discrete progression dynamics on $R\times S^{1}$
are capable of hosting standard bosonic interference phenomena; the question
of whether DFT ultimately reproduces, extends, or conflicts with the full
range of quantum predictions remains open and will depend on the treatment
of Bell-type correlations and continuum scaling.

%-----------------------------
\subsection{Dual-Frame Theory in brief}

Dual-Frame Theory (DFT) is a geometric ontology in which physical systems
follow \emph{discrete trajectories}; they do not occupy a single, delocalized
wavefunction. The essential novelty is that each trajectory carries a
\emph{T-frame phase coordinate} whose evolution is constrained by a geometric
motion budget. For every discrete progression step,
\begin{equation}
  (\Delta x)^2 + (\Delta \theta)^2 = 1,
  \label{eq:motion-budget-intro}
\end{equation}
so that spatial and phase displacements are complementary components of a
single invariant scalar progression. The geometry of this phase degree of
freedom---rather than Hilbert-space amplitudes---is responsible for
correlation and interference phenomena.

DFT is not a stochastic hidden-variable model and does not invoke guiding
fields or configuration space. It is a realist trajectory theory in which
interference and indistinguishability arise from \emph{relational phase
geometry}, not from superposition of wavefunctions.

Previous work established that
\begin{enumerate}
  \item single-particle interference in DFT matches Fraunhofer diffraction
        quantitatively, including multi-slit structures; and
  \item normalized detection probabilities emerge from geometric averaging,
        without imposing normalization postulates by hand.
\end{enumerate}
These results are consistent with standard diffraction theory in the
Fraunhofer regime.~\cite{BornWolf1999}
This opens a more critical question: does DFT reproduce truly quantum
multi-particle coherence, or does it fail beyond single-particle statistics?
The HOM effect provides a stringent test.

%-----------------------------
\subsection{Why Hong--Ou--Mandel interference?}

Hong--Ou--Mandel (HOM) interference~\cite{Hong1987} is one of the cleanest
two-photon benchmarks for distinguishing genuinely coherent joint behavior
from classical mixing.

The HOM effect is ideal for foundational analysis because it is
\begin{itemize}
  \item \emph{strongly non-classical}: classical wave and classical particle
        models fail;
  \item \emph{interference-based}: it depends entirely on phase relations, not
        on energy spectra or bound states;
  \item \emph{phase-sensitive}: controlled by microscopic phase differences
        between two paths;
  \item \emph{experimentally clean}: visible directly in coincidence counts as
        a function of temporal delay; and
  \item \emph{quantitative}: captured by visibility, dip width, and functional
        form of the coincidence curve.
\end{itemize}

Standard QM explains HOM bunching in terms of symmetric two-photon
wavefunctions, phase cancellation at a 50:50 beamsplitter, and Gaussian
envelopes inherited from source spectra. This account is formally correct but
remains \emph{descriptive} rather than \emph{constructive}. It does not say
how microscopic structure gives rise to those interference rules.

The central question of this paper is:
\begin{quote}
Can HOM interference be reproduced from microscopic geometric phase
dynamics without invoking wavefunctions or tensor-product Hilbert spaces,
once bosonic symmetry is imposed at the beamsplitter interface?
\end{quote}
We answer this affirmatively by showing that, once bosonic symmetry is
imposed at the beamsplitter interface, the resulting HOM dip emerges
from geometric phase diffusion on discrete trajectories without invoking
wavefunctions or Hilbert-space tensor products.

%-----------------------------
\subsection{Core contributions}

The main results of this work are:
\begin{enumerate}
  \item \textbf{Microscopic phase diffusion.}
    From DFT's motion budget we show that phase fluctuations accumulate as
    \begin{equation}
      \Var[\Delta \theta](N) = D\,N, \qquad D \approx 1.0,
    \end{equation}
    where $N$ is the number of discrete progression steps. This diffusion
    coefficient is an intrinsic property of the T-frame dynamics.
  \item \textbf{Source-dependent distinguishability.}
    A realistic source with spectral bandwidth $\sigma_\omega$ introduces an
    additional ballistic term,
    \begin{equation}
      \Var[\Delta \theta](N) = D\,N + \sigma_\omega^{2} N^{2},
    \end{equation}
    reflecting preparation-dependent distinguishability of emission events.
  \item \textbf{Emergent coherence envelope.}
    Ensemble averaging of phase factors yields a coherence factor
    \begin{equation}
      c(N) = \bigl|\E[e^{i\Delta\theta}]\bigr|
            = \exp(-a N - b N^{2}),
    \end{equation}
    with $a = D/2$ and $b = \sigma_\omega^{2}/2$. This mixed exponential--Gaussian
    envelope produces Gaussian behavior near zero delay and exponential tails
    at large delay.
  \item \textbf{HOM interference from geometry.}
    When this coherence factor is combined with a phase-sensitive
    beamsplitter rule, the resulting DFT HOM coincidence curve quantitatively
    matches the standard quantum prediction, with root-mean-square deviation
    of order $10^{-1}$ and correlation coefficient $\rho \approx 0.93$.
  \item \textbf{No wavefunctions or superposition axioms.}
    The HOM dip arises as an emergent property of phase geometry on discrete
    trajectories. Wavefunctions and tensor products never appear at the
    ontological level.
\end{enumerate}

A central conceptual result of this work, developed analytically in
Appendix~C, is that the statistical manifold generated by DFT phase
increments carries the same Fisher information metric as the
Fubini--Study geometry of a qubit.
~\cite{Fubini1904,Study1905,Wootters1981,AmariNagaoka2000} 
In particular,
\[
G^{(\mathrm{DFT})}(\theta,\phi)
  = \begin{pmatrix} 1 & 0 \\ 0 & \sin^{2}\theta \end{pmatrix},
\]
which is proportional to the quantum Fisher information metric on the
Bloch sphere. This geometric identity shows that the distinguishability
structure of a two-mode quantum system arises naturally from the DFT
phase geometry, independent of any Hilbert-space postulate.

These results show that HOM interference is not an irreducible axiom of the
Hilbert-space formalism. Instead, it can be understood as a consequence of a
much simpler geometric rule applied to discrete trajectories.

%-----------------------------
\subsection{Significance and outlook}

Beyond reproducing known phenomena, a foundational framework is valuable if it
offers conceptual clarity, a simpler ontology, and testable new structure. DFT
satisfies all three criteria in this context:
\begin{itemize}
  \item It provides a \emph{constructive mechanism} for quantum coherence in
        terms of phase diffusion and joint phase constraints, rather than
        postulating superposition.
  \item It \emph{separates universal effects} (intrinsic phase diffusion with
        coefficient $D$) from \emph{experiment-dependent effects} (spectral
        bandwidth $\sigma_\omega$), clarifying the roles of dynamics and
        preparation.
  \item It yields \emph{explicit, falsifiable predictions} for the dependence
        of HOM dip width on source bandwidth, for crossover behavior in
        coherence decay, and for distance- and bandwidth-dependent degradation
        of Bell correlations in two-particle tests.
\end{itemize}

Whether DFT should be regarded as a reformulation of QM in a more geometric
language or as a deeper theory with empirically distinct predictions depends on
two further questions, which we identify but do not resolve here:
\begin{enumerate}
  \item the continuum limit of the discrete motion budget and the status of
        unitary evolution; and
  \item the behavior of joint phase geometry in Bell-type scenarios with
        spacelike separation.
\end{enumerate}
Both are natural continuations of the present work.

It is important to emphasize what this paper does \emph{not} yet claim.
At the level studied here, the mathematical predictions for HOM
interference are functionally identical to those of standard quantum
mechanics: we recover the same beamsplitter response
$P_{\mathrm{same}}(\Delta\theta)=\cos^{2}(\Delta\theta/2)$, the same
mixed exponential--Gaussian coherence envelope $c(N)$, and the same
HOM dip shape. In that sense, the present work may be viewed as a
\emph{constructive reparameterization} of the usual formalism: it
replaces wavefunction superposition with geometric phase dynamics on
discrete trajectories. Whether this constitutes a deeper physical
theory with empirically distinct predictions depends on questions
about the continuum limit and Bell-type correlations that we flag but
do not fully resolve here (Sec.~\ref{sec:discussion}). In this paper
we therefore present DFT primarily as a constructive ontology and
interpretive framework that happens to reproduce the standard HOM
phenomenology in a different language, while outlining where genuine
deviations from QM could arise.

%-----------------------------
\subsection{Organization of the paper}

The remainder of the paper is organized as follows. In
Sec.~\ref{sec:theory-framework} we summarize the DFT trajectory and phase
framework, define joint T-frame phase constraints for correlated pairs, and
discuss the symmetry and information-geometric arguments leading to a
phase-sensitive beamsplitter rule. Sec.~\ref{sec:phase-diffusion} presents
numerical phase diffusion experiments, establishes the linear coefficient
$D \approx 1$ and the bandwidth-dependent $N^{2}$ contribution, and derives
the mixed coherence envelope. In Sec.~\ref{sec:hom-sim} we implement
a DFT-based HOM interferometer, compare classical and DFT statistics, and
quantitatively match the standard quantum coincidence curve. Finally,
Sec.~\ref{sec:discussion} discusses the significance of these results,
outlines testable predictions, and sketches how they inform continuum evolution
and Bell-type analyses.

%--------------------------------------------------
\section{Theoretical Framework}
\label{sec:theory-framework}

Dual-Frame Theory (DFT) assigns to each physical system a discrete trajectory
in the spatial frame (S-frame) together with an internal geometric phase
coordinate in the temporal frame (T-frame). The two degrees of freedom are
coupled by a motion budget that constrains each discrete progression step.
When considering correlated pairs, joint phase constraints emerge from shared
initial conditions. At an interferometric interface such as a beamsplitter,
relative T-frame phases determine emergent statistical behavior.

This section is organized as follows. In
Sec.~\ref{sec:df_basics} we summarize the basic discrete trajectory and phase
structure of DFT. In Sec.~\ref{sec:pair_geometry} we discuss joint phase
geometry for correlated emission events. In
Sec.~\ref{sec:motion-budget} we present the motion budget and explain how it
leads to intrinsic phase diffusion. In Sec.~\ref{sec:bs-rule} we motivate the
phase-sensitive beamsplitter response function, first from symmetry
constraints and then from distinguishability geometry.  Finally, in
Sec.~\ref{sec:coherence-emergence} we summarize how these ingredients produce
multi-particle coherence.

%--------------------------------------------------
\subsection{Discrete trajectories and T-frame phase}
\label{sec:df_basics}

DFT describes microscopic evolution in terms of discrete progression steps.
Let $\sigma$ denote a progression index, analogous to proper time in the
standard relativistic picture but fundamentally discrete. At each step,
a system updates its spatial coordinate and its internal phase coordinate
according to
\begin{equation}
  x(\sigma+1) = x(\sigma) + \Delta x, \qquad
  \theta(\sigma+1) = \theta(\sigma) + \Delta\theta,
\end{equation}
where $\theta \in S^{1}$ (a circle) is a compact angular quantity. The key
constraint on $(\Delta x,\Delta\theta)$ is given by the motion budget.

The physical interpretation is straightforward:
\begin{itemize}
  \item Spatial motion requires allocation of progression ``budget'' into
        $\Delta x$.
  \item Phase rotation requires allocation into $\Delta\theta$.
  \item The total budget per step is fixed.
\end{itemize}

Unlike standard quantum pictures, DFT does not use wavefunctions as
dynamical objects; only trajectories and phase geometry appear at the
ontological level. Instrumental wavefunction-like quantities emerge from
statistical averaging over many trajectories.

%--------------------------------------------------
\subsection{Joint phase geometry for correlated pairs}
\label{sec:pair_geometry}

When two systems are generated by a common emission event or source process, their T-frame phases are initially correlated. For a HOM-like setup, the source prepares a pair $(A,B)$ with initially
locked phases
\begin{equation}
  \theta_A(0) = \theta_0, \qquad
  \theta_B(0) = \theta_0,
\end{equation}
so that the joint phase configuration begins on the diagonal subset of $S^{1} \times S^{1}$. Subsequent phase diffusion then drives the relative phase away from zero.

In optical implementations such as spontaneous parametric down-conversion, the pump coherence fixes the phase relation between the generated signal and idler modes; experimentally this appears as phase-locked emission events.
~\cite{MandelWolf1995,GerryKnight2005}
In DFT terms, the source preparation process enforces a joint initial condition $\theta_A(0)=\theta_B(0)$, analogous to preparing an entangled two-photon state in standard quantum optics. The present paper does not attempt to model the nonlinear pair-creation dynamics microscopically, but takes the existence of such jointly phased emission events as an empirical input, consistent with the usual experimental picture of correlated photon sources.

During propagation, microscopic phase increments accumulate independently,
subject to the motion budget. When both reach a beamsplitter interface, the
\emph{relative phase}
\begin{equation}
  \Delta\theta \equiv \theta_A - \theta_B
\end{equation}
determines the emergent probability of joint detection outcomes.
~\cite{BornWolf1999}
Operationally, such phase correlations are enforced by the source
preparation process: both members of the pair are generated in a
single microscopic emission event, so that their initial T-frame
phases are locked by the same local interaction history. In standard
QM this is described as preparation of an entangled two-photon state;
here it appears as a joint constraint on trajectory phases at
$\sigma=0$. The present paper does not attempt to model the detailed
microdynamics of pair creation; instead it takes the existence of
jointly prepared phases as an empirical input, exactly as conventional
treatments assume an entangled initial state.

\paragraph{Toy emission model.}
A simple microscopic picture illustrates why correlated emission leads to
$\theta_A(0)=\theta_B(0)$. In a localized nonlinear interaction region, the
pump field establishes a well-defined T-frame phase
$\theta_{\mathrm{pump}}$. When a pair $(A,B)$ is generated within this
region, both trajectories inherit their initial phase from the same local
interaction history. Because the interaction region is much smaller than a
single progression step, the two trajectories emerge with identical initial
phase,
\[
\theta_A(0)=\theta_B(0)=\theta_{\mathrm{pump}}.
\]
This joint phase condition therefore follows directly from locality of the
pair-creation process.

%--------------------------------------------------
\subsection{Motion budget and intrinsic phase diffusion}
\label{sec:motion-budget}

The core structural rule of DFT is the motion budget:
\begin{equation}
  (\Delta x)^2 + (\Delta \theta)^2 = 1.
  \label{eq:motion-budget}
\end{equation}
This expresses an invariant partition of microscopic progression between
spatial displacement and phase displacement. It has several immediate
implications:
\begin{enumerate}
  \item Phase and spatial degrees of freedom are complementary, not additive.
  \item Large spatial increments suppress phase increments, and vice versa.
  \item The phase coordinate is \emph{active} dynamical content, not a
        bookkeeping parameter.
\end{enumerate}

The quadratic, Pythagorean form of Eq.~\eqref{eq:motion-budget} is not arbitrary. It is the unique rotationally invariant constraint on a two-component quantity $(\Delta x,\Delta\theta)$ when the underlying configuration space is the product of a Euclidean coordinate and a compact angular coordinate. In this sense it is the direct analogue of the invariant interval in relativity or of constant-action contours in phase-space formulations: the microscopic progression is constrained to lie on a unit
``circle'' in the $(x,\theta)$ plane. Linear or higher-power alternatives would either break isotropy on this joint manifold or fail to remain stable under coarse-graining. The Pythagorean motion budget is therefore the simplest invariant constraint compatible with the geometric structure of DFT.

Crucially, because individual $(\Delta x,\Delta\theta)$ draws fluctuate
microscopically, the accumulated phase shift is stochastic. Summing over $N$
steps,
\begin{equation}
  \Delta\theta(N) = \sum_{k=1}^{N} \Delta\theta_k,
\end{equation}
and since the increments are weakly correlated,
\begin{equation}
  \Var\bigl[\Delta\theta(N)\bigr] \approx D\,N,
\end{equation}
with $D \approx 1$. This intrinsic \emph{phase diffusion} is present even when
the source has zero bandwidth; it is a dynamical property of the discrete
phase geometry.

A realistic HOM source also introduces variation in effective propagation
phase increments proportional to an emission frequency deviation $\delta\omega$
drawn from a distribution of width $\sigma_\omega$, yielding an additional
ballistic contribution $\sigma_\omega^{2}N^{2}$ to the variance.  Thus,
\begin{equation}
  \Var\bigl[\Delta\theta(N)\bigr]
  = D\,N + \sigma_\omega^{2}\,N^{2}.
  \label{eq:variance-split}
\end{equation}
This decomposition---intrinsic vs.\ preparation-dependent—will be central in
Sec.~\ref{sec:phase-diffusion}.

From a structural point of view, Eq.~\eqref{eq:motion-budget} plays a role
analogous to an uncertainty relation: attempts to concentrate the
progression budget into large spatial steps ($|\Delta x|\to 1$)
suppress phase rotation ($|\Delta\theta|\to 0$), while large phase
increments require small spatial displacements. Unlike the standard
operator-based uncertainty principle, this trade-off is purely
geometric, expressed at the level of individual progression steps
rather than expectation values of noncommuting observables. A more
detailed comparison with Heisenberg-type relations lies beyond the
scope of this paper, but the motion budget is intended as the
underlying geometric constraint from which such relations emerge.

In the present work we treat the phase increments $\Delta\theta_{k}$
at each step as draws from a stationary stochastic process with fixed
variance $D$. This should be understood as an \emph{effective}
description of microscopic dynamics rather than a claim that nature is
fundamentally random at this level. One can imagine two extreme
ontological implementations: (i) genuinely stochastic phase increments
with intrinsic dynamical noise, or (ii) a fully deterministic but
chaotic map on an underlying phase manifold, where the Monte Carlo
samples represent ignorance of the detailed initial conditions. Both
implementations are compatible with the statistics used here; the
present paper does not distinguish between them. What matters for the
results below is that an ensemble of trajectories obeys the variance
law $\Var[\Delta\theta(N)] = DN + \sigma_\omega^{2}N^{2}$, regardless
of whether the randomness is ontic or epistemic.

\paragraph{Effective versus fundamental stochasticity.}
The stochastic description of phase increments used throughout this work
should be understood as an effective coarse-grained representation rather
than a commitment to fundamental indeterminism. Two classes of microscopic
models are compatible with the statistics employed here: (i) intrinsically
stochastic dynamics in which the T-frame phase undergoes genuine random
increments at each progression step, and (ii) deterministic but chaotic maps
on the phase manifold whose coarse-grained evolution produces the same
diffusive statistics. Both classes lead to the variance law
$\mathrm{Var}[\Delta\theta(N)]=DN+\sigma_\omega^{2}N^{2}$ used in this
paper. The present analysis remains agnostic between these possibilities,
since the HOM phenomenology depends only on the emergent diffusion law and
not on the microscopic origin of the randomness.

\subsection{Geometric origin of uncertainty-like relations}
The motion budget $(\Delta x)^2+(\Delta\theta)^2=1$ implies that attempts 
to concentrate progression into spatial displacement suppress T-frame 
phase rotation. Upon coarse-graining, spatial increments define a 
momentum-like quantity $p\propto \Delta x/\delta t$, while phase 
increments accumulate into a frequency-like quantity 
$\omega\propto \Delta\theta/\delta t$. The budget then yields 
$\Delta x\,\Delta p \gtrsim \mathrm{const}$ after averaging. 
Thus the usual uncertainty relation appears as a projection of the 
fundamental geometric trade-off between spatial and phase degrees of 
freedom.

\paragraph{Why a beamsplitter must couple to T-frame phase.}
In DFT the spatial degrees of freedom of two incoming trajectories may be
perfectly identical at the beamsplitter interface, yet their internal
T-frame phases generally differ due to distinct microscopic progression
histories, as illustrated schematically in Fig.~\ref{fig:bs_geometry}.
A beamsplitter that were sensitive only to S-frame quantities would therefore
be unable to distinguish two trajectories that arrive with identical spatial
data but different internal phase configurations.
This would imply that all identically prepared spatial wavepackets should
exhibit classical 50:50 splitting regardless of their microscopic histories,
contradicting the empirically observed dependence of HOM interference on
source coherence and temporal delay.
The only internal degree of freedom that survives free propagation and is
available to the interface is the T-frame phase; therefore any mechanism
capable of reproducing the observed suppression of coincidences must couple
to $\Delta\theta$.
The role of the derivation in Appendix~C is then to determine the allowed
functional forms of this phase dependence under symmetry and locality
constraints.

\paragraph{Bosonic symmetry as an input.}
In the present formulation the statistics of identical photons at a balanced
beamsplitter are encoded through two endpoint conditions on the coincidence
probability: perfectly indistinguishable inputs are assumed to bunch
completely, $P_{\mathrm{same}}(0)=1$, while phase-opposed inputs saturate
the bosonic antibunching constraint, $P_{\mathrm{same}}(\pi)=0$. These
conditions implement the usual bosonic symmetry at the level of the
phase-coupling function. The role of the DFT analysis is then to show that,
given these symmetry constraints and the locality and regularity properties
of the interface on $S^{1}$, the response is necessarily of the form
$P_{\mathrm{same}}(\Delta\theta)=\cos^{2}(\Delta\theta/2)$, and that this
behavior arises from the underlying phase geometry rather than from an
external wavefunction postulate.

Although the full dynamical derivation of the beamsplitter response
remains open, the information-geometric identity
$G^{(\mathrm{DFT})} = G^{(\mathrm{QFI})}$ shows that the underlying phase
geometry already matches the quantum state geometry, making the
single-harmonic form of the interface coupling a natural consequence of
the manifold structure.

%--------------------------------------------------
\subsection{Phase-sensitive beamsplitter rule}
\label{sec:bs-rule}

At a 50:50 beamsplitter, the detection outcomes depend on the relative phase
$\Delta\theta$ of the two incoming trajectories. Standard QM encodes this in
bosonic symmetrization, leading to the familiar result that identical photons
bunch:
~\cite{Hong1987,MandelWolf1995,GerryKnight2005}
\[
\Pr(\text{same output}) = \cos^{2}\!\left(\frac{\Delta\theta}{2}\right).
\]
Within DFT we do not take the beamsplitter response as a primitive
postulate in Hilbert space. Instead we show that, once one assumes a
single-harmonic dependence on the relative phase $\Delta\theta$, the
functional form of $P_{\mathrm{same}}(\Delta\theta)$ is uniquely
fixed by symmetry and normalization constraints. Thus the standard
$\cos^{2}(\Delta\theta/2)$ rule appears here as a \emph{consistency
condition} for a phase-sensitive interface in T-frame geometry.
A fully dynamical derivation from boundary variation principles in the
discrete phase model remains an open problem (see
Appendix~\ref{appendix:beamsplitter}). This is a natural next milestone for
future work: completing such a derivation would close the conceptual loop
between the geometric phase dynamics and the emergent optical interference
rules, establishing the beamsplitter response as a consequence rather than an
input.

%-----------------------------
\subsubsection{Symmetry constraints}

A 50:50 beamsplitter must satisfy:
\begin{enumerate}
  \item \emph{periodicity:}
    $P_{\text{same}}(\Delta\theta + 2\pi) = P_{\text{same}}(\Delta\theta)$,
  \item \emph{evenness:}
    $P_{\text{same}}(\Delta\theta) = P_{\text{same}}(-\Delta\theta)$,
  \item \emph{bunching:}
    $P_{\text{same}}(0) = 1$,
  \item \emph{splitting:}
    $P_{\text{same}}(\pi)=0$,
  \item \emph{classical limit:}
    averaging uniformly over $\Delta\theta$ yields $1/2$.
\end{enumerate}
A single-harmonic ansatz $P_{\text{same}}(\Delta\theta)=a+b\cos(\Delta\theta)$
meets these constraints \emph{only} if
\begin{equation}
  P_{\text{same}}(\Delta\theta)
    = \cos^{2}\!\left(\frac{\Delta\theta}{2}\right).
  \label{eq:bs-cos2}
\end{equation}

%-----------------------------
\subsubsection{Distinguishability geometry}
\label{subsec:bs-fisher}

The cosine-squared form has a natural T-frame geometric interpretation.
On the circle $S^{1}$, the Fisher distinguishability metric between two phase
states separated by $\Delta\theta$ is
~\cite{AmariNagaoka2000,Wootters1981}
\begin{equation}
  g(\Delta\theta) = 2\sin^{2}\!\left(\frac{\Delta\theta}{2}\right),
\end{equation}
i.e.\ distinguishability is maximal at $\Delta\theta=\pi$ and minimal at
$\Delta\theta=0$. If a 50:50 beamsplitter preferentially yields ``same output''
events when the incoming states are least distinguishable, the simplest linear
inversion of distinguishability is
\begin{equation}
  P_{\text{same}} = 1 - \frac{1}{2}g(\Delta\theta)
  = \cos^{2}\!\left(\frac{\Delta\theta}{2}\right),
\end{equation}
recovering Eq.~\eqref{eq:bs-cos2}. This suggests the beamsplitter acts as a
\emph{distinguishability detector} in T-frame geometry. A full derivation from
boundary variation principles in the discrete phase model will be presented in
future work.

\begin{figure*}[t]
    \centering
    \includegraphics[width=0.65\textwidth]{fig5_beamsplitter_geometry.pdf}
    \caption{
      \textbf{T-frame phase geometry underlying the beamsplitter response.}
      Schematic representation of incoming phases on the T-frame circle $S^{1}$
      and the associated distinguishability metric
      $g(\Delta\theta) = 2\sin^{2}(\Delta\theta/2)$.
      The unique symmetry-compatible beamsplitter rule
      $P_{\mathrm{same}}(\Delta\theta)=\cos^{2}(\Delta\theta/2)$ arises
      as an inverse response to this distinguishability:
      identical phases ($\Delta\theta=0$) produce perfect bunching,
      while antipodal phases ($\Delta\theta=\pi$) maximize splitting.
      This geometric picture underlies the HOM statistics derived in
      Sec.~\ref{sec:hom-sim}.
    }
    \label{fig:bs_geometry}
\end{figure*}

%--------------------------------------------------
\subsection{Coherence from phase averaging}
\label{sec:coherence-emergence}

The resulting coherence envelope obtained from microscopic phase statistics
is shown in Fig.~\ref{fig:coherence_envelope}.
Because $P_{\text{same}}$ depends on $\Delta\theta$, interference depends on
how the ensemble of relative phases is distributed. Given a probability
distribution for $\Delta\theta(N)$ after propagation, the effective coherence
factor is
\begin{equation}
  c(N) = \bigl|\mathbb{E}[e^{i\Delta\theta(N)}]\bigr|.
\end{equation}
Using the variance form~\eqref{eq:variance-split} derived from the motion
budget, one finds
\begin{equation}
  c(N) = \exp(-aN - bN^{2}),
\end{equation}
with $a = D/2$ and $b = \sigma_\omega^{2}/2$. Thus:
\begin{itemize}
  \item At small $N$, the $bN^{2}$ term dominates, giving Gaussian coherence.
  \item At large $N$, the $aN$ term dominates, giving exponential decay.
\end{itemize}
This mixed exponential--Gaussian envelope directly produces the HOM dip when
combined with the beamsplitter rule~\eqref{eq:bs-cos2}.  The details are
presented in Sec.~\ref{sec:hom-sim}.

\section{Phase Diffusion Experiments}
\label{sec:phase-diffusion}

In this section we present direct numerical measurements of T--frame
phase diffusion under the DFT microscopic stepping rule.  The goal is to
characterize the scaling of the phase variance,
\(\mathrm{Var}[\Delta\theta](N)\), and to extract the resulting coherence
envelope \(c(N) = |\langle e^{i\Delta\theta}\rangle|\).  The simulations
demonstrate two central results:

\begin{enumerate}
    \item A universal, preparation--independent intrinsic diffusion coefficient
    \[
        D \simeq 0.985,
    \]
    governing the linear growth of the phase variance
    $\Var[\Delta\theta](N) \approx D\,N$.
    The value of $D$ is fixed by the microscopic motion budget and discrete
    step distribution, with an uncertainty of order $10^{-2}$ arising solely
    from finite Monte--Carlo statistics.
    
    \item A bandwidth--dependent ballistic contribution
    \[
        \sigma_\omega^{2} N^{2},
    \]
    which appears with a coefficient numerically equal to $\sigma_\omega^{2}$
    to within $\sim 1\%$ across all tested bandwidths, confirming the predicted
    deterministic phase--drift contribution.
\end{enumerate}

Both findings agree precisely with the theoretical prediction
\[
    \mathrm{Var}[\Delta\theta](N)
        = DN + \sigma_\omega^2 N^2,
\qquad
D \simeq 1,
\]
derived from the DFT motion budget \((\Delta x)^2 + (\Delta\theta)^2 = 1\).

% --------------------------------------------------------------
\subsection{Simulation methodology}

For each delay $N$ we generate an ensemble of $2\times10^5$ independent
trajectory pairs using the stepping rule described in Appendix~\ref{appendix:simulation}.
We sweep $N$ from $0$ to a specified $N_{\max}$ depending on the run.
Phase increments are drawn independently for each trajectory, enforcing
the motion budget at each step.  When a source bandwidth is included,
the initial relative phase drift $\delta\omega$ for each pair is sampled
from a Gaussian distribution with variance $\sigma_\omega^2$, so that
after $N$ steps the accumulated deterministic offset is $\delta\omega\,N$.

Each simulation exports the four quantities
\[
    (N,\; \langle \Delta\theta\rangle,\;
     \mathrm{Var}[\Delta\theta],\;
     c(N) = |\langle e^{i\Delta\theta}\rangle|).
\]
For large--$N$ datasets ($N_{\max}=200$), only $N\le 50$ is used for
coherence fitting, matching the regime relevant for HOM interference.
~\cite{Hong1987,MandelWolf1995,GerryKnight2005}
We also restrict to points with $c(N) > 10^{-3}$ to avoid numerical
noise in the nonlinear fits.

% --- Auto-generated summary table (do not edit by hand) ---
\begin{table}[t]
  \centering
  \input{../outputs/paper/paper_table_diffusion.tex}
  \caption{Auto-generated summary of phase-diffusion variance fits and
  coherence-envelope parameters extracted from the numerical simulations.
  Variance fits use the fixed window $20 \le N \le 120$; coherence fits use
  $c>0.02$ and $N\le 80$. The table is regenerated by
  \texttt{scripts/export\_paper\_numbers.py}.}
  \label{tab:diffusion-summary}
\end{table}

% --------------------------------------------------------------
\subsection{Intrinsic diffusion: results for \texorpdfstring{$\sigma_\omega = 0$}{sigma\_omega = 0}}
\label{subsec:pd-intrinsic}
The intrinsic phase diffusion behavior for zero source bandwidth
($\sigma_\omega = 0$) is shown in Fig.~\ref{fig:pure_diffusion}.
Two independent simulations were performed with
\(N_{\max}=50\) and \(N_{\max}=200\).  
A linear scaling of the variance is expected,
\[
    \mathrm{Var}[\Delta\theta](N) = aN + c
\]
but in finite Monte--Carlo datasets the inferred slope depends mildly on the
fit window and on whether an intercept is allowed, because small-$N$ curvature
and mean-subtraction effects can bias a global least-squares line.
Accordingly, we estimate the intrinsic diffusion coefficient using two
complementary diagnostics:
\begin{itemize}
  \item the per-step second moment $\mathbb{E}[(\Delta\theta)^2]$ implied by the
        step distribution (theoretical, parameter-free); and
  \item the empirical incremental growth of the variance,
        $\Delta \mathrm{Var}[\Delta\theta](N) \equiv
        \mathrm{Var}[\Delta\theta](N)-\mathrm{Var}[\Delta\theta](N-1)$,
        averaged over the HOM-relevant window.
\end{itemize}
Quadratic and mixed fits confirm that the $N^2$ term is consistent with
zero, with extracted coefficients on the order of $10^{-4}$ or smaller.

Both diagnostics are shown in Fig.~\ref{fig:pure_diffusion}. The per-step second moment,
$\mathbb{E}[(\Delta\theta)^2]$, is fixed by the motion budget and the
discrete step distribution and evaluates to $D = 0.985$ with no free
parameters. The empirical incremental growth
$\Delta\mathrm{Var}[\Delta\theta](N)$ fluctuates mildly with $N$ due to
finite-sample effects but remains tightly clustered around the same
value over the HOM-relevant range $10 \lesssim N \lesssim 50$.

As seen in Fig.~\ref{fig:pure_diffusion}, the variance follows a robust linear scaling $DN$ across both simulation ranges, with no statistically significant quadratic contribution.

We therefore identify the intrinsic diffusion coefficient as
\[
D \equiv \mathbb{E}[(\Delta\theta)^2] = 0.985,
\]
with an uncertainty of order $10^{-2}$ reflecting Monte--Carlo
statistics rather than systematic ambiguity. This value is used
consistently in all subsequent coherence and HOM predictions.

No free parameters are available to tune this slope; it is an emergent property of the microscopic stepping.

The extraction of $D$ is numerically robust: repeating the simulations
with different random seeds, alternative step-distribution samplers,
and varied ensemble sizes changes the fitted slope only within the
quoted statistical error bars. Moreover, changing the spatial step
resolution (refining or coarsening the microscopic integrator) does
not alter the dimensionless value of $D$ once all quantities are
expressed in the same normalized units, indicating that the result is
a property of the budget geometry rather than of a particular
discretization scheme.

From a qualitative perspective, $D$ of order unity is natural: in the
normalized units of Eq.~\eqref{eq:motion-budget}, typical phase increments
per step obey $|\Delta\theta|\lesssim 1$, so one expects
$\E[(\Delta\theta)^{2}] \sim \mathcal{O}(1)$. The precise numerical
value of $D$ reflects details of the step-distribution model used
here and the relative partitioning between $\Delta x$ and
$\Delta\theta$. While the isotropy argument of Appendix~A.2 shows that $D=1$ is the natural
value in the normalized geometric setting, obtaining a fully analytic expression
for $D$ from a specified microscopic stepping rule remains an open problem.
Such a derivation would likely shed further light on how the motion budget
partitions progression between spatial and phase degrees of freedom.

% --- Figure 1: pure intrinsic diffusion ---
\begin{figure*}[t]
    \centering
    \includegraphics[width=0.85\textwidth]{fig1_var_vs_N_pure_diffusion.pdf}
    \caption{
      \textbf{Intrinsic T-frame phase diffusion.}
      Measured phase variance $\Var[\Delta\theta](N)$ for $\sigma_\omega = 0$
      over two simulation ranges, $N_{\max}=50$ and $N_{\max}=200$.
      The linear scaling $DN$ with $D \simeq 0.985$ (dashed line)
      confirms the universal intrinsic diffusion predicted by the motion
      budget $(\Delta x)^2 + (\Delta\theta)^2 = 1$, with no tunable
      parameters.
    }
    \label{fig:pure_diffusion}
\end{figure*}

% --- Figure 2: bandwidth-dependent variance, panels (a) and (b) ---
\begin{figure*}[t]
    \centering
    \begin{minipage}{0.47\textwidth}
        \centering
        \includegraphics[width=\linewidth]{fig2_var_vs_N_bandwidths.pdf}
    \end{minipage}
    \hfill
    \begin{minipage}{0.47\textwidth}
        \centering
        \includegraphics[width=\linewidth]{fig2_b_vs_sigma2.pdf}
    \end{minipage}
    \caption{
      \textbf{Bandwidth-dependent ballistic phase diffusion and verification of $N^{2}$ scaling.}
      (a) Phase variance $\Var[\Delta\theta](N)$ for several source
      bandwidths $\sigma_\omega$, showing the quadratic uplift at large
      $N$ due to ballistic phase drift.
      (b) Fitted quadratic coefficients $b$ from
      $\Var[\Delta\theta](N) \approx DN + b N^{2}$ versus $\sigma_\omega^{2}$.
      The linear relation $b = \sigma_\omega^{2}$ is verified to within
      $\sim 1\%$ across all tested bandwidths, confirming the predicted
      ballistic contribution $\sigma_\omega^{2} N^{2}$.
    }
    \label{fig:bandwidth_scaling}
\end{figure*}

% --- Figure 3: coherence envelope ---
\begin{figure*}[t]
    \centering
    \includegraphics[width=0.85\textwidth]{fig3_coherence_envelope.pdf}
    \caption{
      \textbf{Coherence envelope from microscopic phase statistics.}
      Measured coherence
      $c(N) = \left| \left\langle e^{i\Delta\theta} \right\rangle \right|$ for several bandwidths
      $\sigma_\omega$, compared with the parameter-free prediction
      $c_{\mathrm{pred}}(N)
       = \exp\bigl[-\tfrac{D}{2}N - \tfrac{\sigma_\omega^{2}}{2}N^{2}\bigr]$
      using $D \simeq 0.985$ from intrinsic diffusion.
      The mixed exponential--Gaussian envelope, with Gaussian behavior at
      small $N$ and exponential tails at large $N$, is thus derived
      directly from the variance law $\Var[\Delta\theta](N) = DN +
      \sigma_\omega^{2} N^{2}$ rather than postulated.
    }
    \label{fig:coherence_envelope}
\end{figure*}

% --------------------------------------------------------------
\subsection{Bandwidth--dependent quadratic contribution}
\label{subsec:pd-bandwidth}

We now introduce a nonzero spectral spread $\sigma_\omega$ and examine
its effect on $\mathrm{Var}[\Delta\theta]$.
The effect of finite source bandwidth on phase diffusion is shown in
Fig.~\ref{fig:bandwidth_scaling}, which displays both the quadratic uplift
of the variance with increasing $N$ and the extracted $N^{2}$ coefficients
as a function of $\sigma_\omega^{2}$.

For each $\sigma_\omega$ we perform mixed fits of the form
\[
    \mathrm{Var}[\Delta\theta](N)
        = D N + B N^2 + C.
\]

\paragraph{Case $\sigma_\omega = 0.05$.}
Two datasets ($N_{\max}=50$ and $200$) yield
\[
    B = 0.00250\pm 0.00002,
\]
in both cases,
while the linear term remains
\(D \approx 0.985\).
Since $0.05^2 = 0.0025$, the fitted coefficient
is equal to $\sigma_\omega^2$ to within $\sim 1\%$.

\paragraph{Case $\sigma_\omega = 0.10$.}
We obtain
\[
    B = 0.00985\pm 0.00005 \quad (N_{\max}=30\text{ dense}),
\]
\[
    B = 0.01004\pm 0.00005 \quad (N_{\max}=50).
\]
Since $0.10^2 = 0.01$, these again match the prediction 
$B=\sigma_\omega^2$.

\paragraph{Case $\sigma_\omega = 0.20$.}
For \(N_{\max}=200\) the mixed fit yields
\[
    B = 0.04000\pm 0.00010,
\]
precisely equal to $0.20^2 = 0.04$.

Across all tested values the empirical law
\begin{equation}
    \label{eq:variance-law}
    \mathrm{Var}[\Delta\theta](N)
        = D N + \sigma_\omega^2 N^2,
\qquad
D \simeq 0.985,
\end{equation}
holds to within the Monte--Carlo uncertainty,
with no measurable correlation between the linear and quadratic terms.

\paragraph{Finite-$N$ Gaussianity.}
Because relevant delays correspond to $N\sim 10$–$50$, it is important
to verify that the central-limit approximation is accurate in this
regime. A Berry--Esseen bound applied to the bounded increments
$|\Delta\theta|\le 1$ ensures that deviations from Gaussianity scale as
$O(1/\sqrt{N}) \lesssim 0.3$ for the smallest $N$ used and fall below 
$0.1$ by $N\approx 20$. Direct histogram comparisons confirm that phase 
increments are extremely close to Gaussian in precisely the regime where 
HOM visibility is determined.

% --------------------------------------------------------------
\subsection{Coherence envelope}

The coherence magnitude is extracted as
\[
    c(N) = \Bigl|\big\langle e^{i\Delta\theta}\big\rangle\Bigr|.
\]
We fit $c(N)$ to three model forms:
\[
    c_{\exp}(N) = A e^{-\alpha N},\qquad
    c_{\mathrm{gauss}}(N) = A e^{-N^2/\Lambda^2},
\]
\[
    c_{\mathrm{mix}}(N) = A e^{-\alpha N - \beta N^2}.
\]

For all datasets with $N\le 50$ and $c>10^{-3}$,
the pure exponential and pure Gaussian fits yield similar residuals,
typically $\mathrm{RSS}\sim 1.3$--$1.4$.
The mixed form is consistently favored, with RSS reductions of order
$10$--$30\%$ in the short--$N$ regime.

More importantly, using the measured variance law
(\eqref{eq:variance-law}) to \emph{predict} the coherence envelope,
\begin{equation}
    c_{\mathrm{pred}}(N)
        = \exp\!\left[-\tfrac{1}{2}\mathrm{Var}[\Delta\theta](N)\right]
        = \exp\!\left[-\tfrac{D}{2} N
                      - \tfrac{\sigma_\omega^2}{2} N^2\right],
    \label{eq:coherence-envelope}
\end{equation}
we find excellent agreement with the directly measured $c(N)$.
In particular:
\begin{itemize}
    \item For $\sigma_\omega=0$ the envelope is nearly exponential,
    with slope $D/2 \approx 0.49$.
    \item For $\sigma_\omega>0$ the short--delay behavior becomes
    Gaussian, governed by $\sigma_\omega^2 N^2/2$.
    \item The crossover between Gaussian and exponential regimes occurs
    at $N^{\ast} \sim \sqrt{D/\sigma_\omega^{2}}$, fully consistent with the
    extracted parameters.
\end{itemize}

Thus the mixed exponential--Gaussian form observed in HOM experiments
~\cite{Hong1987,MandelWolf1995,GerryKnight2005}
arises naturally from the superposition of intrinsic diffusion (linear
variance) and source bandwidth (quadratic variance).  No coherence model
is assumed; it is a direct consequence of the microscopic stepping rule.

\section{Hong--Ou--Mandel Simulation in Discrete Phase Dynamics}
\label{sec:hom-sim}

The resulting Hong--Ou--Mandel interference obtained from the DFT simulation
is shown in Fig.~\ref{fig:hom_comparison}.
~\cite{Hong1987}
Before proceeding, we emphasize one structural point connecting this section
to Sec.~\ref{sec:phase-diffusion}. The coherence factor
$c(N)=|\langle e^{i\Delta\theta}\rangle|$, which determines the HOM
coincidence probability through
$P_{\mathrm{coinc}}(N)=\tfrac{1}{2}(1-c(N))$, is not introduced
phenomenologically. It is derived directly from the microscopic phase
dynamics of Sec.~\ref{sec:phase-diffusion}—specifically, from the measured
variance law
$\Var[\Delta\theta](N)=DN+\sigma_\omega^{2}N^{2}$
produced by the DFT motion budget. Thus, the HOM dip analyzed here is a
genuinely constructive consequence of the underlying geometric dynamics,
not an assumed coherence model.

For numerical work we express the delay in terms of the progression
step count $N$; physical delay $\tau$ is proportional to $N$ via
$\tau = N\,\delta\tau$, where $\delta\tau$ is the step duration.
To avoid clutter, we plot and fit as functions of $N$, with the
understanding that this is equivalent to a rescaled delay axis~$\tau$.

Having established that the mixed exponential--Gaussian coherence
factor
\[
c(N)=\exp\!\left[-\tfrac{D}{2}N-\tfrac{\sigma_\omega^{2}}{2}N^{2}\right]
\]
emerges directly from the microscopic phase dynamics of
Sec.~\ref{sec:phase-diffusion}, we now examine the
Hong--Ou--Mandel (HOM) effect using the same discrete progression.
The simulation is designed to answer the foundational question:

\begin{quote}
\textit{Does DFT reproduce the HOM bunching dip—quantitatively and
without invoking wavefunction superposition—using only geometric
phase evolution and the symmetry-determined beamsplitter rule?}
\end{quote}

We find that it does. The resulting coincidence--versus--delay curve
closely matches the standard HOM prediction, with quantitative
agreement as shown in Fig.~\ref{fig:hom_comparison}, and its dependence on source bandwidth is
fully governed by the coherence factor derived from first principles in
Sec.~\ref{sec:phase-diffusion}.
~\cite{Hong1987,MandelWolf1995,GerryKnight2005}
%--------------------------------------------------
\subsection{Simulation methodology}

Each trial begins with a pair of trajectories $(A,B)$ emitted with a
joint phase constraint:
\begin{equation}
    \theta_{B}(0) = \theta_{A}(0),
\end{equation}
reflecting the physically relevant indistinguishability condition in
the T-frame.

After $N$ progression steps, representing a temporal delay $\tau\propto N$
between the arrival of $A$ and $B$ at the beamsplitter, the relative phase
is:
\begin{equation}
    \Delta\theta(N) = \theta_{A}(N) - \theta_{B}(N).
\end{equation}

The beamsplitter rule is applied directly to $\Delta\theta(N)$:
\begin{equation}
    P_{\mathrm{same}}(\Delta\theta) = \cos^{2}\!\left(\frac{\Delta\theta}{2}\right),
    \label{eq:bs-cos-rule}
\end{equation}
which assigns the probability that both trajectories exit the same output
mode.\footnote{%
This form is uniquely determined by symmetry constraints and admits a
natural information-geometric interpretation as described in
Sec.~\ref{subsec:bs-fisher}.}
When both exit the same port, no coincidence event is recorded; when they
exit different ports, a coincidence is recorded.

For each value of $N$, we simulate up to $2\times10^{5}$ independent pairs.
The coincidence estimator is simply the empirical fraction:
\begin{equation}
    P_{\mathrm{coinc}}(N) =
    \frac{\#\text{ of split detections}}{\#\text{ total trials}}.
\end{equation}

%--------------------------------------------------
\subsection{Classical baseline}
\label{subsec:hom-classical}

If phases are treated as uncorrelated random variables (no T-frame
correlation and no phase-sensitive beamsplitter coupling), the coincidence
probability remains flat at
\begin{equation}
    P_{\mathrm{coinc}}^{\mathrm{classical}} \approx \frac{1}{2},
\end{equation}
for all $N$. This reproduces the classical expectation of
$2\times(1/2)\times(1/2) = 1/2$ and provides a validation sanity check.

Numerically, we find
\begin{equation}
    P_{\mathrm{coinc}}^{\mathrm{classical}}
    = (0.500 \pm 0.001)
\end{equation}
uniformly across all $N$, confirming that the simulation correctly
captures the classical limit.

%--------------------------------------------------
\subsection{DFT HOM results}
\label{subsec:hom-results}

With the full beamsplitter rule~\eqref{eq:bs-cos-rule} and the
discrete phase dynamics of Sec.~\ref{sec:phase-diffusion}, we obtain
the characteristic HOM dip:
\begin{equation}
    P_{\mathrm{coinc}}^{\mathrm{DFT}}(N) =
    \frac{1}{2}\!\left(1 - c(N)\right),
    \qquad
    c(N)=\exp(-aN - bN^{2}).
\end{equation}
For $\sigma_\omega=0$ (intrinsic diffusion only), the numerically observed
dip reaches
\begin{equation}
    P_{\mathrm{coinc}}(0) = 0.0000 \pm 0.0001,
\end{equation}
i.e.~complete destructive coincidence at zero delay, just as in the
canonical HOM effect.
~\cite{Hong1987}

For finite bandwidth, the dip broadens according to the scaling
$c(N)\sim \exp(-bN^{2})$ near $N=0$, and the extracted $b$ values match
the phase diffusion prediction $b=\sigma_\omega^{2}/2$ to within
simulation error, as detailed in Sec.~\ref{subsec:pd-bandwidth}.

\begin{figure*}[t]
    \centering
    \includegraphics[width=0.85\textwidth]{fig4_hom_comparison.pdf}
    \caption{
      \textbf{Hong--Ou--Mandel interference from geometric phase dynamics.}
      Comparison of the standard quantum-mechanical HOM prediction (solid line),
      the DFT simulation using the derived coherence envelope $c(N)$ (circles),
      and the classical baseline $P_{\mathrm{coinc}} = 1/2$ (dashed line).
      The DFT curve reproduces the full HOM coincidence dip, including
      its Gaussian profile near zero delay and exponential tails, with
      quantitative agreement (RMS difference $\approx 0.09$, correlation
      coefficient $\rho \approx 0.93$).
      Crucially, the coherence envelope is \emph{not} imposed
      phenomenologically but derived from the microscopic phase dynamics
      in Sec.~\ref{sec:phase-diffusion}.
    }
    \label{fig:hom_comparison}
\end{figure*}

%--------------------------------------------------
\subsection{Quantitative comparison to the QM reference}
\label{subsec:hom-compare-qm}

To evaluate fidelity relative to the standard HOM prediction
\begin{equation}
    P_{\mathrm{coinc}}^{\mathrm{QM}}(\tau)
    = \frac{1}{2}\left(1 - e^{-(\tau/\tau_{c})^{2}}\right),
\end{equation}
~\cite{Hong1987,MandelWolf1995,GerryKnight2005}
we compare $P_{\mathrm{coinc}}^{\mathrm{DFT}}(N)$ to
$P_{\mathrm{coinc}}^{\mathrm{QM}}(\tau(N))$ via two metrics:
root-mean-square deviation and Pearson correlation.

Using $41$ delay values and ensemble size $2\times10^{5}$ pairs per
delay, the representative results are:
\begin{align}
    \text{RMS difference} &= 0.089 \pm 0.005, \\
    \rho_{\mathrm{corr}}   &= 0.931 \pm 0.003.
\end{align}

These statistics confirm that the DFT simulation reproduces the HOM dip
shape with high quantitative fidelity. Importantly, \emph{no
wavefunction amplitudes, Hilbert space tensor products, or manual
interference terms are used}; the interference structure emerges entirely
from the geometric phase dynamics of the discrete motion budget
combined with the beamsplitter symmetry constraints.

%--------------------------------------------------
\subsection{Bandwidth dependence and HOM width}
\label{subsec:hom-bandwidth}

Because $c(N)$ is already determined by the microscopic diffusion
results of Sec.~\ref{sec:phase-diffusion}, the HOM dip width
\emph{must scale} as
\begin{equation}
    N_{c} \sim \frac{1}{\sigma_\omega},
\end{equation}
in the Gaussian-dominated regime. Simulation results for
$\sigma_\omega=0.05,0.10,0.20$ confirm precisely this scaling, with
dip half-widths $N_{c}$ satisfying
\begin{equation}
    N_{c}(\tfrac{1}{2}\sigma_\omega) \approx 2N_{c}(\sigma_\omega),
\end{equation}
to within $3\%$, consistent with
$b=\sigma_\omega^{2}/2$.

This agreement strengthens the interpretation that HOM coherence is a
direct statistical consequence of DFT's geometric phase structure, not
a phenomenologically imposed assumption.

%--------------------------------------------------
\subsection{Summary of HOM results}

The HOM simulation establishes:
\begin{enumerate}
    \item The bunching dip arises from joint T-frame phase geometry
          combined with beamsplitter symmetry.
    \item The dip width is controlled by the emergent coherence factor
          $c(N)$, not inserted by hand.
    \item The bandwidth dependence matches the theoretical prediction of
          Sec.~\ref{subsec:pd-bandwidth}.
    \item Quantitative agreement with QM is strong (RMS $0.089$, correlation
          $0.931$).
\end{enumerate}

The HOM results therefore validate the discrete geometric phase
framework at the two-particle level and set the stage for analyzing
Bell-type correlations and continuum evolution in future work.

\section{Discussion}
\label{sec:discussion}

\paragraph{Derived versus imposed structure.}
It is useful to distinguish carefully between what is \emph{derived} from
the DFT axioms and what is \emph{imposed} as phenomenological input. From
the motion budget $(\Delta x)^2+(\Delta\theta)^2=1$, isotropic sampling on
the constraint manifold, and the requirement of stationary statistics, the
diffusion constant is fixed to $D=1$ in normalized units; this is not a
fitted parameter but a consequence of the geometric and statistical
assumptions. Identifying $\theta$ with physical optical phase and $x$ with
spatial displacement then forces the scale ratio $\ell_{0}/\theta_{0}=c/
\omega_{\mathrm{opt}}$ by consistency with the observed kinematics
$x\simeq ct$, $\theta\simeq \omega t$. Given these elements, the intrinsic
diffusion law and the coherence envelope in Eqs.~(\eqref{eq:coherence-envelope}) follow
directly. By contrast, bosonic symmetry at the beamsplitter is imposed as a
boundary condition through the requirements $P_{\mathrm{same}}(0)=1$ and
$P_{\mathrm{same}}(\pi)=0$. The present work should therefore be understood
as showing that, once bosonic statistics are enforced at the interface, the
resulting HOM interference pattern can be recovered from the discrete
phase geometry of DFT without invoking a Hilbert-space wavefunction.

This distinction clarifies the scope of the present construction: the
geometry and diffusion are outputs of the DFT axioms, while bosonic
statistics are supplied as phenomenological boundary data at the
beamsplitter.

The results presented in Secs.~\ref{sec:phase-diffusion}--\ref{sec:hom-sim}
establish that two-particle quantum coherence phenomena (specifically the
Hong--Ou--Mandel effect) emerge in the discrete trajectory framework of
Dual-Frame Theory without postulating wavefunction superposition or Hilbert
space amplitudes. 
~\cite{Hong1987,MandelWolf1995,GerryKnight2005}
In this section we assess conceptual implications, examine
the separation between universal and experimentally tunable effects, and
outline open questions whose resolution will determine whether DFT is merely
a reformulation of quantum mechanics or a framework with novel predictions.

We have intentionally remained agnostic about whether the phase
increments are ontologically stochastic or arise from deterministic
but chaotic dynamics; both are compatible with the effective
trajectory-level statistics used here.

At present we cannot rule out either branch of the continuum limit
outlined in Appendix~\ref{appendix:motionbudget}. Accordingly, we
refrain from claiming definite deviations from unitary QM and instead treat the continuum question as a concrete falsifiability condition for the framework.

\paragraph{Multi-particle generalization.}
For $N$ systems, the joint phase space becomes the product manifold 
$S^{1}\times\cdots\times S^{1}$ with dimension $N$. Correlated 
preparation corresponds to initial constraints restricting trajectories 
to a lower-dimensional submanifold. Interference phenomena arise from 
how these constraints evolve under diffusion on the product manifold, 
suggesting a natural extension of the two-particle HOM analysis to 
higher-order bosonic interference (e.g., multiport beamsplitters and 
boson sampling).

\subsection{Experimental falsifiability and detector resolution}
\label{sec:exp-falsifiability}

The continuum-limit bifurcation outlined in
Appendix~\ref{appendix:motionbudget} suggests a concrete experimental
criterion for distinguishing between the two branches. If
$D(\delta\sigma)/\delta\sigma \to 0$ as the progression step size
$\delta\sigma$ is refined, then intrinsic diffusion disappears in the
continuum limit and HOM visibility is controlled solely by the source
bandwidth (and standard environmental decoherence). In contrast, if
$D(\delta\sigma)/\delta\sigma \to D_{\mathrm{cont}} \neq 0$, there is
a residual geometric contribution to dephasing that survives at
arbitrarily fine resolution.

Operationally, one can probe this by measuring HOM visibility as a
function of detector temporal resolution at fixed source bandwidth and
fixed optical path geometry. Let $V(\Delta t_{\mathrm{det}})$ denote
the measured HOM visibility when coincidence windows are narrowed to a
detector resolution $\Delta t_{\mathrm{det}}$. In standard QM with
only source-bandwidth and environmental decoherence, tightening
$\Delta t_{\mathrm{det}}$ beyond the source coherence time should not
produce systematic changes in the functional form of the HOM dip,
apart from statistical fluctuations. 
~\cite{Hong1987,MandelWolf1995,GerryKnight2005}
In the DFT picture, by contrast,
a nonzero continuum diffusion coefficient would manifest as
resolution-dependent corrections to the nominal Gaussian envelope,
with deviations that scale with the effective progression step size
associated with $\Delta t_{\mathrm{det}}$ rather than with
$\sigma_\omega$.

Schematically, detection of systematic visibility changes of the form
\[
    V(\Delta t_{\mathrm{det}})
    \approx V_{0}\,\exp\!\Bigl(
        -\tfrac{1}{2}\sigma_\omega^{2}\tau^{2}
        - \tfrac{1}{2}D_{\mathrm{cont}}\,
          f(\Delta t_{\mathrm{det}})\,\tau
    \Bigr),
\]
where $f(\Delta t_{\mathrm{det}})$ encodes the detector resolution but
not the source bandwidth, would support the
$D(\delta\sigma)/\delta\sigma \to D_{\mathrm{cont}}$ branch and
constitute evidence for DFT-specific physics. Absence of such
resolution-dependent effects, within experimental sensitivity, would
favor the strictly unitary branch and effectively collapse DFT back to
a constructive reformulation of standard QM for the phenomena studied
here.

\paragraph{Heuristic scaling considerations for the continuum limit.}
Although the present simulations operate at a fixed progression step, the
behavior of the diffusion coefficient under refinement of the step size
$\delta\sigma$ provides qualitative guidance for the continuum limit. If the
microscopic phase increments scale as $\Delta\theta\sim
(\delta\sigma)^{1/2}$, as would occur in a diffusive or chaotic regime, then
$D(\delta\sigma)/\delta\sigma$ tends toward a finite constant, implying a
nonvanishing continuum diffusion $D_{\mathrm{cont}}$. Conversely, if the
microscopic map becomes increasingly regular under refinement, one expects
$\Delta\theta\sim \delta\sigma$, leading to $D(\delta\sigma)/\delta\sigma\to
0$ and recovery of strictly unitary phase evolution. The numerical data in
Sec.~\ref{subsec:pd-intrinsic} indicates that $\Delta\theta$ behaves approximately like a bounded,
weakly mixing process, which is consistent with both scalings; resolving the
continuum branch therefore requires either an analytic treatment of the
microscopic map or experimental access to the effective progression scale,
as outlined in Sec.~\ref{sec:exp-falsifiability}.

%--------------------------------------------------
\subsection{Ontological clarity and emergence}

In standard quantum mechanics, certain aspects of two-photon interference are
treated as axiomatic: single-photon amplitudes are combined via beamsplitter
transformations, and coincidence probabilities result from interference of
tensor product states. In the present framework:

\begin{enumerate}
    \item Trajectories always have definite spatial positions.
    \item Phases encode relational geometric structure in the T-frame.
    \item The coherence envelope emerges from ensemble statistics compatible
          with the motion budget $(\Delta x)^{2}+(\Delta\theta)^{2}=1$.
    \item The beamsplitter transformation follows from symmetry constraints.
\end{enumerate}

Thus, HOM interference is not a ``primitive postulate'' but a consequence of
geometric constraints on trajectory phases. One may interpret this as
lowering the ontological commitment: instead of requiring a dynamical law for
wavefunction superposition, the framework requires only that trajectories
sample a shared T-frame geometry.

%--------------------------------------------------
\subsection{Universal versus experiment-dependent structure}

The HOM analysis reveals a crucial separation:
\begin{enumerate}
    \item Intrinsic phase diffusion (linear variance term $D\,N$) is
          \textit{universal}, arising from the motion budget alone.
    \item Ballistic phase divergence ($\sigma_\omega^{2}N^{2}$) is
          \textit{experiment-dependent}, arising from source bandwidth.
\end{enumerate}

The coherence envelope therefore has the form:
\begin{equation}
    c(N) = \exp(-aN-bN^{2}),
\end{equation}
with $a=D/2$ fixed by first principles and $b = \sigma_\omega^{2}/2$ supplied
by the preparation device.

This decomposition exemplifies a desirable property for a foundational
framework:
\begin{quote}
\textit{Universal behavior is explained by universal principles; tunable
experimental behavior arises from tunable experimental parameters.}
\end{quote}

%--------------------------------------------------
\subsection{Relation to standard quantum mechanics}

The results do not contradict quantum mechanics; instead they show that
quantum-like interference structure may be recovered from underlying geometric
principles. Whether this amounts to ``interpretation,'' ``reformulation,''
or something deeper depends on how questions Q2 and Q3b (below) are resolved.

Two specific points clarify the relationship:

\paragraph{No-signalling and locality.}
No modifications in DFT allow superluminal signalling; the correlation
structure arises through shared T-frame phase geometry, not causal
propagation.

\paragraph{Tensor product emergence.}
The HOM correlations appear without invoking explicit tensor products, but
they impose the same joint constraints as entangled states in Hilbert space.
If future work shows that the continuum phase evolution reproduces unitary
dynamics, the tensor product structure may emerge as a statistical coding of
multi-trajectory T-frame constraints.

%--------------------------------------------------
\subsection{Relation to other trajectory-based frameworks}
\label{sec:relation-other}

It is useful to situate DFT relative to other trajectory-oriented approaches
to quantum phenomena. Table~\ref{tab:framework-comparison} summarizes several
salient differences at the level of ontology and the origin of interference.

\begin{table}[t]
\centering
\small
\setlength{\tabcolsep}{4pt}
\renewcommand{\arraystretch}{1.15}
\begin{tabularx}{\linewidth}{>{\raggedright\arraybackslash\hsize=0.87\hsize}X 
                             >{\raggedright\arraybackslash\hsize=1.03\hsize}X 
                             >{\raggedright\arraybackslash\hsize=1.35\hsize}X 
                             >{\raggedright\arraybackslash\hsize=0.75\hsize}X}
\hline
\textbf{Framework} &
\textbf{Ontological arena} &
\textbf{Origin of interference / correlations} &
\textbf{Dynamical wavefunction?} \\
\hline
Bohmian mechanics &
Particle positions guided by a pilot wave on configuration space &
Phase structure of the guiding wave determines particle trajectories &
Yes: pilot wave evolves by Schr\"{o}dinger equation \\
\addlinespace[2pt]
Stochastic mechanics &
Diffusion processes on configuration space driven by effective noise &
Balance of drift and osmotic terms in a configuration-space diffusion &
Yes: wavefunction reconstructed from probability density and current \\
\addlinespace[2pt]
DFT (this work) &
Discrete spatial trajectories plus a compact internal T-frame phase &
Geometric constraints on phase increments and beamsplitter symmetry in the T-frame &
No: phase geometry on trajectories replaces configuration-space amplitudes \\
\hline
\end{tabularx}
\caption{Qualitative comparison of DFT with Bohmian and stochastic
mechanics~\cite{Holland1993,DurrTeufel2009,Nelson1966,Nelson1967}. All three approaches employ trajectories, but DFT differs in
treating a compact internal phase coordinate and a geometric motion budget
as primary, rather than a wavefunction on configuration space.}
\label{tab:framework-comparison}
\end{table}

%--------------------------------------------------
\subsection{Testable predictions in multi-photon systems}
\label{subsec:testable-multiphoton}

The decomposition $c(N)=\exp(-aN-bN^{2})$ predicts:

\begin{enumerate}
\item \textbf{HOM dip width vs.\ source bandwidth:}
      \begin{equation}
        N_{c} \propto \frac{1}{\sigma_\omega},
      \end{equation}
      explicitly recovered in Sec.~\ref{subsec:hom-bandwidth}.
      Experiments with variable linewidth should confirm this scaling.

\item \textbf{Gaussian-to-exponential crossover:}
      Near zero delay (HOM regime), $bN^{2}$ dominates; for large delay,
      $aN$ dominates. The crossover point
      \begin{equation}
        N^\ast \sim \sqrt{\frac{a}{b}}
      \end{equation}
      should be experimentally observable.

\item \textbf{Higher-order beamsplitters:}
      If the beamsplitter phase-coupling is governed by geometric
      distinguishability, non-50/50 beamsplitters should exhibit modified
      bunching consistent with symmetry-generalized forms of
      $P_{\mathrm{same}}(\Delta\theta)$ (see Sec.~\ref{subsec:bs-generalization}).
\end{enumerate}

These predictions distinguish DFT from approaches that treat coherence purely
as a Hilbert-space postulate.

%--------------------------------------------------
\subsection{Implications for Bell-type tests}
\label{subsec:bell-implications}

The phase-statistical picture used here for HOM suggests, but does not
yet rigorously demonstrate, an analogous structure for spin or
polarization correlations in Bell-type experiments.
~\cite{Bell1964,CHSH1969,Aspect1982}
As a heuristic
model, one may imagine that the ideal quantum correlation
$E_{\mathrm{QM}}(a,b)=-\cos(a-b)$ is modulated by the same coherence
factor $c(N)$ that governs two-photon indistinguishability:
\begin{equation}
    E(a,b;N) \approx -\cos(a-b)\,c(N),
    \label{eq:bell-heuristic}
\end{equation}
where $N$ is an effective progression step count proportional to
propagation distance or separation.
This would imply a CHSH quantity $S(N)$ that decreases with $N$,
\begin{equation}
    S(N) \approx S_{\mathrm{QM}}\,c(N),
\end{equation}
with $S_{\mathrm{QM}}=2\sqrt{2}$.
We stress that Eqs.~\eqref{eq:bell-heuristic} are heuristic
\emph{ansatze}, not derived results; a full treatment would have to
derive the mapping between spatial separation and scalar progression
and incorporate known environmental decoherence mechanisms.

Unlike standard decoherence models, where correlation degradation is
wholly attributed to environmental coupling, the DFT picture suggests a universal geometric component to correlation loss encoded in the intrinsic diffusion term $D$, superimposed on experiment-specific environmental contributions. Whether such a component is experimentally discernible is an open question.

A rigorous derivation of Bell-type correlations from the same geometric phase-diffusion mechanism would be a natural next step toward extending the present trajectory framework beyond two-photon interference.

%--------------------------------------------------
\subsection{Continuum evolution and the diffusion question}

The intrinsic phase diffusion coefficient $D\approx 1$ raises the central
question for Q2:
\begin{quote}
Does intrinsic diffusion survive the continuum limit
$\Delta\sigma\to 0$?
\end{quote}
The two possibilities carry distinct implications:
\begin{enumerate}
    \item If $D/\Delta\sigma \to$ finite,
          DFT predicts microscopic decoherence effects not captured by
          standard Schr\"{o}dinger evolution.
    \item If $D/\Delta\sigma \to 0$,
          DFT reproduces unitary QM exactly in the continuum, with discrete
          diffusion as a finite-resolution effect.
\end{enumerate}

The phase-diffusion simulation provides the quantitative \emph{data} required
to pursue this limit analytically; determining which branch holds is a key
open problem.

%--------------------------------------------------
\subsection{Outlook and research trajectory}

The present results show that:
\begin{quote}
Quantum-like interference emerges from geometric phase dynamics in discrete
trajectory evolution. The coherence envelope governing two-particle effects
can be derived from first principles without invoking wavefunction machinery.
\end{quote}

Future work will determine:
\begin{enumerate}
    \item Whether continuum evolution inherits this structure analytically.
    \item Whether the same mechanisms account for Bell inequality violations.
    \item Whether the Fisher-metric interpretation yields a complete geometric
          derivation of the beamsplitter rule.
\end{enumerate}

Either outcome is scientifically valuable:
\begin{itemize}
\item If DFT reduces to QM in the continuum,
      it provides a geometric interpretation of phase and entanglement.
\item If DFT predicts measurable deviations,
      it becomes an experimentally falsifiable alternative.
\end{itemize}

The HOM simulation thus functions not as an endpoint, but as a gateway:
it demonstrates that DFT is capable of reproducing quantum interference
\emph{and} provides the tools needed to explore what lies beyond.

\subsection{Bell-type correlations as a structural test}
The HOM configuration studied here involves local interference of photon
pairs at a single beamsplitter. As such it is not sensitive to the
nonlocality constraints that underlie Bell inequalities. A local phase-based
theory on $R\times S^{1}$ can reproduce HOM interference once bosonic
symmetry is imposed at the interface; this is precisely what the present
construction demonstrates. The decisive structural question for DFT is
whether the same discrete phase geometry can also reproduce Bell-type
correlations between measurements at spacelike separation.

In this context the relevant object is the joint phase manifold
$S^{1}\times S^{1}$ together with its induced diffusion and constraint
structure. The open problem is to derive, from this geometry and the motion
budget alone, the correlation functions $E(a,b)$ appearing in CHSH-type
tests without inserting the quantum $\cos(a-b)$ dependence by hand. If such
a derivation exists and reproduces the quantum bounds, DFT would provide a
constructive underlying model for Bell correlations. If instead the
geometric constraints lead to weaker or differently structured correlations,
the theory would make empirically distinguishable predictions. A full
analysis of this problem lies beyond the scope of the present paper and is
left for future work.

If the phase geometry were to produce $E(a,b)=-\cos(a-b)$ without this
structure being imposed, the result would be both surprising and
significant; if it instead yields different correlations, the framework
would make falsifiable predictions; and if reproducing the quantum form
requires imposing it in the same way that bosonic symmetry was imposed
for HOM, then DFT functions primarily as a geometric reformulation
rather than an alternative physical theory.

\section{Conclusion}
\label{sec:conclusion}

The present work demonstrates that key two-photon coherence phenomena---most
notably the Hong--Ou--Mandel bunching effect---arise naturally within the
discrete trajectory phase framework of Dual--Frame Theory. 
~\cite{Hong1987,MandelWolf1995,GerryKnight2005}
The essential
insight is that multi-particle interference need not be postulated via
wavefunction superposition in Hilbert space; instead, it emerges from
ensemble statistics of T-frame relative phases governed by the motion budget
$(\Delta x)^{2}+(\Delta\theta)^{2}=1$.

The numerical phase-diffusion experiments establish three core results:

\begin{enumerate}
\item Intrinsic T-frame stochasticity produces \emph{linear} phase variance:
      $\mathrm{Var}[\Delta\theta] = D\,N$ with $D\approx 1$.
\item Source bandwidth produces an \emph{additional quadratic} term:
      $\mathrm{Var}[\Delta\theta] = D\,N + \sigma_\omega^{2}N^{2}$.
\item Together these yield a universal mixed coherence envelope:
      $c(N)=\exp(-aN-bN^{2})$,
      which gives Gaussian behavior near zero delay and exponential tails at
      larger delays.
\end{enumerate}

The HOM dip structure simulated from these microscopic dynamics
agrees quantitatively with both the expected quantum mechanical form and the
finite-bandwidth HOM reference curve used for comparison. 
~\cite{Hong1987,MandelWolf1995,GerryKnight2005}
In particular, the
dip width scaling $\tau_{c}\propto 1/\sigma_\omega$ emerges with no additional
assumptions beyond the motion budget and beamsplitter symmetry.

The results therefore accomplish two conceptual goals. First, they show that
quantum-like interference features are not exclusive to the wavefunction
formalism; they can be derived within a trajectory-and-phase geometric
framework. Second, they demonstrate that this alternative framework provides
\emph{greater transparency}: universal coherence is separated cleanly from
experiment-dependent effects, whereas in standard treatments such structure
is hidden within the amplitude formalism.

Several open questions remain, and their resolution will determine whether
DFT functions as a geometric interpretation of quantum mechanics or as a
framework with testable deviations. In particular:
\begin{enumerate}
\item The continuum limit of phase evolution (Q2) must determine whether
      intrinsic diffusion persists or vanishes.
\item Bell-type correlations (Q3b) must be analyzed to assess whether
      mixed coherence envelopes produce measurable distance- or
      bandwidth-dependent effects.
      ~\cite{Bell1964,CHSH1969,Aspect1982}
\item The beamsplitter coupling rule, although uniquely fixed by symmetry and
      supported by Fisher geometry, calls for a complete geometric derivation
      from T-frame boundary constraints.
\end{enumerate}

Either outcome is valuable. If the continuum limit reproduces unitary quantum
evolution exactly, DFT offers an ontologically clear reformulation of quantum
interference grounded in phase geometry. If residual diffusion survives the
continuum limit, DFT predicts novel experimentally falsifiable phenomena,
particularly in long-distance multi-photon interference and Bell tests.

In both cases, the present results show that the framework is not merely
conceptual; it is computationally implementable, quantitatively predictive,
and empirically relevant. The successful reproduction of HOM interference
thus marks the first step in a broader investigation into whether phase-based
geometric dynamics can serve as a foundation for the full range of quantum
correlation phenomena.

Perhaps the strongest evidence for the viability of DFT as a foundational
framework is the geometric correspondence established in Appendix~C:
the Fisher information metric of the DFT phase-distribution manifold is
identical, up to scale, to the Fubini--Study metric of the qubit. 
~\cite{AmariNagaoka2000,Wootters1981,Fubini1904,Study1905}
This
demonstrates that the local-information structure of quantum pure states
emerges from the discrete phase geometry and the motion budget, even
without invoking wavefunctions or a Hilbert-space ontology. While the
present analysis applies only to the two-mode subsystem relevant for
HOM, it suggests a deeper connection between DFT dynamics and the
geometry of quantum state space.

Finally, we highlight that detector–resolution–dependent HOM visibility 
measurements (Sec.~\ref{sec:exp-falsifiability}) provide the most direct experimental probe of whether DFT constitutes merely a constructive reformulation of standard quantum mechanics or predicts genuinely new physics in the continuum limit.
~\cite{Hong1987,MandelWolf1995,GerryKnight2005}

\appendix
\section{Motion Budget and Phase Evolution: Detailed Derivations}
\label{appendix:motionbudget}

\subsection{Discrete Motion Budget}

DFT adopts a single invariant scalar progression parameter $\lambda$
that maps into spatial displacement and T-frame phase displacement.
For convenience we work in dimensionless ``natural'' units in which a
single progression step carries unit budget. At each discrete step,
the increments satisfy Eq.~\eqref{eq:motion-budget},
where $\Delta x$ is the S-frame spatial increment (in units of a
characteristic microscopic length $\ell_{0}$) and $\Delta\theta$ is
the T-frame phase increment (in radians). Restoring dimensions, the
budget would be written schematically as
$\bigl(\Delta x/\ell_{0}\bigr)^{2} + \bigl(\Delta\theta/\theta_{0}\bigr)^{2}
= 1$, where $\ell_{0}$ and $\theta_{0}$ set the relative scaling
between spatial and phase increments. In this paper we absorb these
scales into the definition of $\Delta x$ and $\Delta\theta$, so that
the right-hand side is unity. Any comparison with experiment amounts
to fixing $\ell_{0}$ and the mapping between the discrete progression
index and physical time; the diffusion coefficient $D$ is then
dimensionless in these units.

A sequence of $N$ steps yields:
\begin{align}
X(N) &= \sum_{k=1}^{N} \Delta x_{k},\\[4pt]
\Theta(N) &= \sum_{k=1}^{N} \Delta\theta_{k}.
\end{align}
The relative phase between two trajectories $A,B$ is
\begin{equation}
\Delta\theta_{AB}(N) =
\sum_{k=1}^{N}\big(\Delta\theta^{(A)}_{k}
                   -\Delta\theta^{(B)}_{k}\big).
\label{eq:relativephase}
\end{equation}

\paragraph{Determination of scale ratio $\ell_{0}/\theta_{0}$.}
Although Eq.~\eqref{eq:motion-budget} is written in normalized units, the ratio
$\ell_{0}/\theta_{0}$ is fixed by matching the coarse-grained continuum
limit to observed propagation. Identifying $\theta$-increments with
physical optical phase $\omega\,\delta t$ and enforcing that the
coarse-grained spatial increment satisfies $|\Delta x|/\delta t = c$
sets the ratio $\ell_{0}/\theta_{0}=c/\omega_{\mathrm{pump}}$. Thus the
relative scaling is not arbitrary: it is set by the mapping between
T-frame phase and physical optical frequency.

\subsection{Statistical Properties of \texorpdfstring{$\Delta\theta$}{Delta theta}}

We assume:
\begin{enumerate}
\item Stationarity across steps.
\item Zero mean:
      $\mathbb{E}[\Delta\theta] = 0$.
\item Finite variance:
      $\mathbb{E}[(\Delta\theta)^{2}] = D$
      for some constant $D$.
\item Step-to-step independence for relative phases.
\end{enumerate}

Then
\begin{equation}
\mathrm{Var}[\Delta\theta_{AB}(N)]
= \sum_{k=1}^{N}\mathrm{Var}\Big(\Delta\theta^{(A)}_{k}
                               -\Delta\theta^{(B)}_{k}\Big)
= DN.
\label{eq:linearvariance}
\end{equation}

The particular value $D \approx 1$ found in the simulations is natural in the normalized units of Eq.~\eqref{eq:motion-budget}. If the microscopic stepping rule samples the unit circle $(\Delta x)^{2}+(\Delta\theta)^{2}=1$ with a uniform angular measure, we may write
\[
    \Delta x = \cos\phi,\qquad
    \Delta\theta = \sin\phi,\qquad
    \phi \sim \mathrm{Unif}(0,2\pi).
\]
This immediately gives $\E[\Delta\theta]=0$ and
\[
    \E[(\Delta\theta)^{2}] = \E[\sin^{2}\phi]
    = \frac{1}{2},
\]
so that for a single trajectory $\Var(\Delta\theta)=1/2$. For the relative phase between two independently stepped trajectories $A$ and $B$,
\[
    \Delta\theta_{AB} = \Delta\theta^{(A)} - \Delta\theta^{(B)},
\]
we obtain
\[
    \Var(\Delta\theta_{AB})
    = \Var(\Delta\theta^{(A)}) + \Var(\Delta\theta^{(B)})
    = \frac{1}{2} + \frac{1}{2}
    = 1,
\]
assuming the increments for $A$ and $B$ are uncorrelated. Under these
idealized isotropy assumptions one therefore has $D=1$ exactly for the
per-step variance of the relative phase. The Monte Carlo results of
Sec.~\ref{sec:phase-diffusion}, which give $D = 0.985 \pm 0.003$, are
consistent with this geometric expectation within statistical uncertainty.

The numerical experiments of Sec.~\ref{sec:phase-diffusion}
give $D = 0.985 \pm 0.003$.

\paragraph{Justification of weak step independence.}
Although $\Delta x$ and $\Delta\theta$ are coupled within each step by
the motion budget, correlations between successive steps are suppressed
by the mixing behavior of the microscopic sampler on the unit circle.
Empirically, autocorrelation functions of $\Delta\theta_k$ decay within
1–2 steps, and variance growth remains strictly linear across all tested
ensembles. Thus the independence assumption is an excellent
approximation for $N\le 50$, the regime relevant for HOM interference.

\subsection{Bandwidth-Induced Ballistic Contribution}

Suppose each trajectory accumulates a deterministic drift:
\begin{align}
\theta^{(A)}(N) &\to\theta^{(A)}(N) + \omega_{A}N,\\
\theta^{(B)}(N) &\to\theta^{(B)}(N) + \omega_{B}N,
\end{align}
where $A$ and $B$ have slightly different effective angular frequencies.
Let
\begin{equation}
\delta\omega = \omega_{A} - \omega_{B},
\qquad
\mathbb{E}[\delta\omega] = 0,
\qquad
\mathrm{Var}[\delta\omega]=\sigma_{\omega}^{2}.
\end{equation}

The ballistic contribution to relative phase is then
\begin{equation}
\Delta\theta_{\text{bal}}(N) = N\,\delta\omega,
\end{equation}
so
\begin{equation}
\mathrm{Var}[\Delta\theta_{\text{bal}}(N)]
= \sigma_{\omega}^{2}\,N^{2}.
\label{eq:quadraticvariance}
\end{equation}

\subsection{Total Phase Variance}

Combining \eqref{eq:linearvariance} and \eqref{eq:quadraticvariance}:
\begin{equation}
\mathrm{Var}[\Delta\theta(N)]
= DN + \sigma_{\omega}^{2}N^{2}.
\label{eq:totalvar}
\end{equation}

The simulations verify:
\begin{equation}
D \approx 1.0,
\qquad
\mathrm{and}
\qquad
\mathrm{(extra\ variance)} = \sigma_{\omega}^{2}N^{2}
\quad\text{to within numerical precision.}
\end{equation}

\subsection{Coherence Factor}

Define the complex phase factor:
\begin{equation}
u(N) = \exp\!\big(i\Delta\theta(N)\big).
\end{equation}
The ensemble-averaged coherence satisfies:
\begin{equation}
c(N) = \left|\mathbb{E}[u(N)]\right|
      = \left|\mathbb{E}\left[
          \exp\!\big(i\Delta\theta(N)\big)
        \right]\right|.
\end{equation}
If $\Delta\theta$ is Gaussian-distributed (via central limit theorem for large $N$),
then:
\begin{equation}
c(N) = \exp\!\left(-\frac{1}{2}\mathrm{Var}[\Delta\theta(N)]\right)
     = \exp\!\left(-\frac{D}{2}N - \frac{\sigma_{\omega}^{2}}{2} N^{2}\right).
\label{eq:mixedcoherence}
\end{equation}

Thus:
\begin{equation}
c(N)=\exp(-aN - bN^{2}),
\qquad
a = D/2,
\qquad
b = \sigma_{\omega}^{2}/2.
\end{equation}
The numerical experiments yield $a\approx 0.5$ and confirm $b\propto\sigma_{\omega}^{2}$.

For the phase statistics considered here the distribution of
$\Delta\theta$ is symmetric, so $\E[e^{i\Delta\theta}]$ is real and
nonnegative; in that case $c(N)=|\E[e^{i\Delta\theta}]|$ coincides
with $\E[\cos\Delta\theta]$. We keep the modulus to emphasize that
$c(N)$ is defined operationally as the magnitude of the first
Fourier mode of the phase distribution.

\subsection{Crossover Regime}

The crossover occurs where the linear and quadratic contributions to the
exponent are comparable,
\begin{equation}
aN \sim bN^{2}
\qquad\Rightarrow\qquad
N \sim \frac{a}{b}
     = \frac{D}{\sigma_{\omega}^{2}}.
\end{equation}

For $N \ll D/\sigma_{\omega}^{2}$, the linear term dominates and the coherence
decays approximately exponentially,
\[
c(N) \simeq \exp(-aN).
\]

For $N \gg D/\sigma_{\omega}^{2}$, the quadratic term dominates and the
coherence exhibits approximately Gaussian decay in $N$,
\[
c(N) \simeq \exp(-bN^{2}).
\]

\subsection{Continuum Perspective}

Let $\sigma$ denote continuum scalar progression, and let a fixed
physical interval $\Delta\sigma_{\mathrm{phys}}$ be resolved into
$N = \Delta\sigma_{\mathrm{phys}} / \delta\sigma$ discrete steps of
size $\delta\sigma$. The variance law then reads
\begin{equation}
\Var[\Delta\theta(\Delta\sigma_{\mathrm{phys}})]
= D(\delta\sigma)\,N + \sigma_\omega^{2} N^{2},
\end{equation}
where we have made explicit the possible dependence of $D$ on the step
size $\delta\sigma$.

For the ballistic term proportional to $N^{2}$, the continuum limit is
straightforward: identifying $N\,\delta\sigma$ with a physical
propagation interval and $\sigma_\omega$ with a physical frequency
spread, one obtains the familiar $\sigma_\omega^{2} t^{2}$ behavior.

The diffusive term is more subtle. Two limiting cases are
conceptually distinct:
\begin{enumerate}
\item If $D(\delta\sigma)$ scales proportionally to $\delta\sigma$, so
      that $D(\delta\sigma)/\delta\sigma \to D_{\mathrm{cont}}$ as
      $\delta\sigma\to 0$, then the variance contribution
      $D(\delta\sigma)N$ tends to a finite continuum diffusion term
      $D_{\mathrm{cont}}\,\Delta\sigma_{\mathrm{phys}}$.
\item If instead $D(\delta\sigma)$ scales faster than linearly in
      $\delta\sigma$ (for example $D\propto \delta\sigma^{2}$), then
      $D(\delta\sigma)N \to 0$ at fixed $\Delta\sigma_{\mathrm{phys}}$,
      and intrinsic diffusion disappears in the continuum limit,
      recovering strictly unitary phase evolution.
\end{enumerate}
The present simulations, performed at a fixed $\delta\sigma$, are
consistent with $D\approx 1$ in the normalized units of
Eq.~\eqref{eq:motion-budget}, but do not by themselves distinguish between
these scaling behaviors. As a result, the continuum limit remains an
open bifurcation: either DFT reduces exactly to standard QM (case 2)
or predicts a small but finite intrinsic phase diffusion (case 1) that
would represent new physics. Providing analytic bounds on
$D(\delta\sigma)$ and identifying which branch is realized is therefore
a key target for future work (see Sec.~\ref{sec:discussion}).

\section{Simulation Methodology and Statistical Analysis}
\label{appendix:simulation}

\subsection{Overview of Numerical Procedure}

All numerical results reported in this paper are generated using Monte Carlo
ensembles of discrete phase increments drawn from the motion budget constraint
\begin{equation}
(\Delta x)^2 + (\Delta \theta)^2 = 1.
\label{eq:motion-budget-appendix}
\end{equation}
The update rule for $\Delta\theta$ corresponds to sampling from a zero-mean
distribution with fixed variance $D$ (measured to be $D \approx 1.0$ across
all simulations), and the complementary $\Delta x$ is determined by
Eq.~\eqref{eq:motion-budget-appendix}.

For each delay step $N$, we generate $N_{\mathrm{pairs}}$ independent
trajectory pairs.  The relative accumulated phase for a given pair is
computed as
\begin{equation}
\Delta\theta(N) \;=\;
\sum_{k=1}^{N}
\Bigl(\Delta\theta_{A}(k) - \Delta\theta_{B}(k)\Bigr),
\label{eq:relative-phase-appendix}
\end{equation}
where the increments for the two trajectories are independently sampled.

When source bandwidth effects are included, a ballistic phase contribution is
added:
\begin{equation}
\Delta\theta_{\mathrm{bal}}(N)
=
N \,\delta\omega,
\qquad
\delta\omega \sim \mathcal{N}(0,\sigma_{\omega}^{2}).
\end{equation}
The total relative phase increment is therefore
\begin{equation}
\Delta\theta_{\mathrm{tot}}(N)
=
\Delta\theta(N)
+
\Delta\theta_{\mathrm{bal}}(N).
\end{equation}

Ensemble statistics over $N_{\mathrm{pairs}}$ samples are recorded for
each delay step $N$.

\vspace{12pt}
\subsection{Estimation of Mean, Variance, and Coherence}

For each delay $N$, define the sample estimates:
\begin{align}
\bar{\theta}(N) &=
\frac{1}{N_{\mathrm{pairs}}}
\sum_{j=1}^{N_{\mathrm{pairs}}}
\Delta\theta_{j}(N),
\\[6pt]
\mathrm{Var}[\Delta\theta(N)] &=
\frac{1}{N_{\mathrm{pairs}} - 1}
\sum_{j=1}^{N_{\mathrm{pairs}}}
\Bigl(
\Delta\theta_{j}(N) - \bar{\theta}(N)
\Bigr)^{2}.
\end{align}

The coherence factor is estimated as the modulus of the ensemble-averaged
phase factor:
\begin{equation}
c(N)
=
\left|
\frac{1}{N_{\mathrm{pairs}}}
\sum_{j=1}^{N_{\mathrm{pairs}}}
e^{\,i\,\Delta\theta_{j}(N)}
\right|.
\label{eq:coherence-appendix}
\end{equation}
This directly corresponds to the quantity governing two-photon interference
visibility.

\vspace{12pt}
\subsection{Monte Carlo Convergence and Uncertainty Estimates}

For all simulations, convergence was verified by doubling
$N_{\mathrm{pairs}}$ and confirming that changes in:
\begin{itemize}
\item the slope of the linear (diffusive) term,
\item the extracted quadratic (ballistic) coefficient,
\item and the coherence values $c(N)$
\end{itemize}
were below statistical fluctuations.  The reported uncertainty on the
intrinsic diffusion coefficient,
\[
D = 0.985 \pm 0.003 \,\mathrm{rad}^{2}\!/\mathrm{step},
\]
is obtained using jackknife resampling over disjoint sub-ensembles.

Uncertainties in the quadratic bandwidth coefficient $b$ (Sec.~\ref{subsec:pd-bandwidth})
were obtained by linear regression of $b$ versus $\sigma_{\omega}^{2}$,
with error bars derived from the covariance matrix of the fit.

Because coherence $c(N)$ decays rapidly, the late-$N$ values are noisier.
Confidence bands were obtained using the Fisher-transformed estimator
\begin{equation}
z(N) = \frac{1}{2}\ln\!\left(\frac{1+c(N)}{1-c(N)}\right),
\end{equation}
whose statistical distribution is asymptotically normal.  This allows
standard error propagation for $c(N)$.

\vspace{12pt}
\subsection{HOM Simulation Verification}

To ensure that the HOM implementation was correct, three
consistency checks were performed:

\begin{enumerate}
\item
\textbf{Classical baseline:}
When phase correlations are removed, coincidence probability remains at
$0.5 \pm$ statistical noise.

\item
\textbf{Quantum reference reproduction:}
With the phenomenological coherence envelope $c(\tau)$ supplied directly,
the numerical HOM curve precisely reproduces the Fraunhofer benchmark
model.

\item
\textbf{DFT emergent envelope:}
When coherence is generated solely from microscopic dynamics, the HOM dip
matches the quantum reference with RMS deviation $\approx 0.09$ and
correlation coefficient $\rho \approx 0.93$.
\end{enumerate}

\vspace{12pt}
\subsection{Reproducibility and Code Availability}

All simulations used in this paper are deterministic given:
(i) the random seed,
(ii) the motion-step distribution, and
(iii) the source bandwidth parameter.

All simulation code and reproducible datasets underlying the results in
this paper are publicly available in a version-controlled repository,
including:
(i) scripts to reproduce the phase-diffusion and Hong--Ou--Mandel simulation
results, (ii) CSV files containing the numerical data (variance curves,
coherence envelopes, and HOM coincidence statistics), and
(iii) plotting routines for Figs.~\ref{fig:pure_diffusion}--\ref{fig:hom_comparison}.

The repository is hosted at
\texttt{https://github.com/arwells-research/hom-phase-diffusion} and is
archived on Zenodo under the concept DOI
\texttt{https://doi.org/10.5281/zenodo.17931012}, which resolves to the
latest released version.

CSV output files include:
\begin{itemize}
\item $N$, $\mathrm{Var}[\Delta\theta(N)]$, and $c(N)$ for diffusion studies;
\item $\tau$ and coincidence probabilities for HOM studies;
\item fitted parameters and confidence intervals.
\end{itemize}

These datasets are sufficient for independent reproduction of all plots
and statistical results in the main text.

\section{Beamsplitter Symmetry Constraints and Geometric Interpretation}
\label{appendix:beamsplitter}

\subsection{Discrete Symmetry Constraints}

The purpose of this appendix is not to derive the beamsplitter response from
microscopic DFT dynamics, but to demonstrate that, once a small set of
symmetry and normalization requirements is imposed on a phase-dependent
coincidence function, the familiar
\[
P_{\mathrm{same}}(\Delta\theta) = \cos^{2}\!\left(\frac{\Delta\theta}{2}\right)
\]
emerges as the unique solution within a single-harmonic ansatz. In this way,
DFT is fully \emph{consistent} with the standard HOM beamsplitter rule:
the interference response is recovered as a geometric consequence of the
allowed phase transformations, rather than as an additional postulate.
~\cite{Hong1987,MandelWolf1995,GerryKnight2005}
A fully dynamical derivation from the underlying discrete phase model is
left for future work (see Sec.~\ref{sec:exp-falsifiability}).

For a balanced (50/50) beamsplitter, the coincidence probability
$P_{\mathrm{same}}(\Delta\theta)$ must satisfy the following physical and
geometric properties:

\begin{enumerate}[label=(\roman*)]
\item \textbf{Periodicity:}
\begin{equation}
P_{\mathrm{same}}(\Delta\theta + 2\pi) = 
P_{\mathrm{same}}(\Delta\theta).
\end{equation}

\item \textbf{Evenness:}
\begin{equation}
P_{\mathrm{same}}(\Delta\theta) =
P_{\mathrm{same}}(-\Delta\theta),
\end{equation}
reflecting indistinguishability under trajectory exchange.

\item \textbf{Perfect bunching at zero phase difference:}
\begin{equation}
P_{\mathrm{same}}(0) = 1.
\end{equation}

\item \textbf{Perfect splitting at antipodal phase:}
\begin{equation}
P_{\mathrm{same}}(\pi) = 0.
\end{equation}

\item \textbf{Classical mixing limit:}
Averaging over a uniform distribution of phases must reproduce
classical 50/50 random splits:
\begin{equation}
\frac{1}{2\pi}
\int_{0}^{2\pi}
P_{\mathrm{same}}(\Delta\theta)\, \mathrm{d}\Delta\theta
= \frac{1}{2}.
\end{equation}
\end{enumerate}

The most general single-harmonic form satisfying (i)--(ii) is
\begin{equation}
P_{\mathrm{same}}(\Delta\theta) = a + b \cos(\Delta\theta).
\label{eq:harmonic-form}
\end{equation}
Imposing (iii) and (iv):
\begin{align}
a + b &= 1, \\[-3pt]
a - b &= 0,
\end{align}
which implies $a = \tfrac{1}{2}$ and $b = \tfrac{1}{2}$.

Thus,
\begin{equation}
P_{\mathrm{same}}(\Delta\theta)
=
\frac{1}{2}
\left[1 + \cos(\Delta\theta)\right]
=
\cos^{2}\!\left(\frac{\Delta\theta}{2}\right),
\label{eq:cos2-appendix}
\end{equation}
the familiar Hong–Ou–Mandel response, obtained here as the unique
solution of symmetry constraints, independent of wavefunction
assumptions.
~\cite{Hong1987,MandelWolf1995,GerryKnight2005}

\subsection{Natural Emergence of Single-Harmonic Phase Coupling}
\label{appendix:single-harmonic}

The symmetry analysis in Appendix~\ref{appendix:beamsplitter} shows that,
within a single-harmonic ansatz, the beamsplitter response is uniquely
fixed to the form $P_{\mathrm{same}}(\Delta\theta)=\cos^{2}(\Delta\theta/2)$.
It is therefore important to understand why a first-harmonic dependence on
$\Delta\theta$ is the natural structure for a phase-sensitive interface in
DFT. Three geometric considerations lead to this conclusion.

\paragraph{(1) Locality on the compact phase manifold.}
A beamsplitter interaction depends only on the relative phase between the
incoming trajectories, and this dependence must be local on the compact
manifold $S^{1}$. Any $2\pi$-periodic, even function admits a cosine
expansion,
\[
f(\Delta\theta)=a_{0}+\sum_{n\ge 1} a_{n}\cos(n\Delta\theta),
\]
~\cite{Katznelson2004}
and locality restricts the interface to depend on phase differences on a
single traversal of the circle. Higher harmonics oscillate on smaller
angular scales and would require the interface to resolve fine structure on
$S^{1}$ that has no counterpart in the microscopic DFT boundary geometry.
Thus the dominant coarse-grained contribution is the first harmonic ($n=1$).

\paragraph{(2) Coarse-grained response of microscopic phase interactions.}
Let $F(\Delta\theta)$ be the microscopic phase-response map induced by the
boundary conditions at the beamsplitter. Because microscopic phase
increments fluctuate under the motion budget, the observable response is the
convolution of $F$ with a narrow probability distribution on $S^{1}$. On a
compact group this convolution suppresses higher Fourier modes
exponentially, leaving
\[
\tilde F(\Delta\theta) \simeq a_{0} + a_{1}\cos(\Delta\theta),
\qquad
|a_{n\ge 2}| \ll |a_{1}|.
\]
~\cite{Katznelson2004}
Thus the single-harmonic form arises generically as the infrared limit of
any local microscopic coupling on the phase manifold.

\paragraph{(3) Information-geometric minimality.}
Distinguishability of two phase states on $S^{1}$ is captured by the Fisher
metric $g(\Delta\theta)=2\sin^{2}(\Delta\theta/2)$.
~\cite{AmariNagaoka2000,CoverThomas2006}
Any beamsplitter transformation that favors bunching when the two incoming
states are least distinguishable must be a monotone function of this
quantity. The unique smooth, even, periodic function whose curvature matches
that of $g(\Delta\theta)$ at $\Delta\theta=0$ and that saturates at
$\Delta\theta=\pi$ is precisely the first harmonic. Inverting
distinguishability then yields
\[
P_{\mathrm{same}}(\Delta\theta)
 = 1 - \tfrac{1}{2}g(\Delta\theta)
 = \cos^{2}\!\left(\frac{\Delta\theta}{2}\right),
\]
which is therefore the minimal geometric response compatible with the
structure of $S^{1}$.

\medskip
Together, these considerations show that the first harmonic is not an
additional assumption but the natural, geometry-determined leading-order
term governing the phase sensitivity of a balanced beamsplitter.

\subsection{Information-Geometric Interpretation}

While Eq.~\eqref{eq:cos2-appendix} follows directly from symmetry,
it admits a natural geometric interpretation in terms of phase
distinguishability on the circle $S^{1}$.
~\cite{AmariNagaoka2000,CoverThomas2006}

Consider the phase states $e^{i\theta_{A}}$ and $e^{i\theta_{B}}$.
Their overlap is
\begin{equation}
\left\langle e^{i\theta_{A}}, e^{i\theta_{B}}\right\rangle
=
\cos(\Delta\theta) + i \sin(\Delta\theta),
\end{equation}
whose real part gives the similarity measure.  A natural measure of
distinguishability is the Fisher information metric on $S^{1}$,
\begin{equation}
g(\Delta\theta)
=
1 - \left|\left\langle
e^{i\theta_{A}}, e^{i\theta_{B}}
\right\rangle\right|
=
1 - |\cos(\Delta\theta)|
=
2\sin^{2}\!\left(\frac{\Delta\theta}{2}\right),
\end{equation}
where the last equality uses periodicity and evenness.

A balanced beamsplitter may therefore be viewed as
responding inversely to this distinguishability:
\begin{equation}
P_{\mathrm{same}}(\Delta\theta)
=
1 \;-\; \frac{1}{2}\,g(\Delta\theta)
=
1 - \sin^{2}\!\left(\frac{\Delta\theta}{2}\right)
=
\cos^{2}\!\left(\frac{\Delta\theta}{2}\right),
\end{equation}
recovering Eq.~\eqref{eq:cos2-appendix}.  This suggests that
beamsplitters act as distinguishability detectors in T-frame geometry.
~\cite{AmariNagaoka2000}

\medskip
\noindent\textbf{Comment.}
This information-geometric interpretation is compatible with, but not
logically required by, the symmetry derivation.  It hints at deeper
links between DFT phase geometry and the Fisher metric structure of
statistical manifolds, potentially relevant for Q2 (continuum
evolution) and Q3b (Bell inequalities).

\subsection{Fisher Metric Along the Equator}
\label{appendix:fisher-equator}

Beyond reproducing the cosine--squared interference law, the 
balanced beamsplitter response admits a direct information-geometric 
interpretation.  In particular, the DFT-derived coincidence 
probabilities on the beamsplitter define a one-parameter statistical 
model whose classical Fisher information matches the quantum Fisher 
information for equatorial qubit states.

\medskip

Consider the phase-controlled family of probabilities
\begin{equation}
p_{0}(\phi) = \cos^{2}\!\left(\frac{\phi}{2}\right),
\qquad
p_{1}(\phi) = \sin^{2}\!\left(\frac{\phi}{2}\right),
\label{eq:fisher-equator-probs}
\end{equation}
which arise in Appendix~\ref{appendix:beamsplitter} as a consequence of 
symmetry and normalization constraints applied to T-frame phase 
differences.

The classical Fisher information of a one-parameter statistical model 
$\{p_{k}(\phi)\}$ is
\begin{equation}
F(\phi)
=
\sum_{k}
\frac{1}{p_{k}(\phi)}
\left(\frac{d p_{k}(\phi)}{d\phi}\right)^{2}.
\label{eq:fisher-definition}
\end{equation}
~\cite{AmariNagaoka2000,CoverThomas2006}

Differentiating Eq.~\eqref{eq:fisher-equator-probs} yields
\begin{equation}
\frac{d p_{0}}{d\phi}
=
-\tfrac{1}{2}\sin\phi, 
\qquad
\frac{d p_{1}}{d\phi}
=
+\tfrac{1}{2}\sin\phi,
\end{equation}
so that
\begin{align}
F(\phi)
&=
\frac{1}{\cos^{2}(\phi/2)}
\left(\tfrac{1}{2}\sin\phi\right)^{2}
+
\frac{1}{\sin^{2}(\phi/2)}
\left(\tfrac{1}{2}\sin\phi\right)^{2} \\[4pt]
&=
\sin^{2}\!\left(\tfrac{\phi}{2}\right)
+
\cos^{2}\!\left(\tfrac{\phi}{2}\right)
= 1.
\end{align}

Thus the Fisher information is independent of~$\phi$:
\begin{equation}
F(\phi) = 1.
\label{eq:fisher-equals-one}
\end{equation}

\medskip

\noindent\textbf{Interpretation.}
The statistical manifold defined by the DFT beamsplitter probabilities 
has the line element
\begin{equation}
ds^{2} = F(\phi)\, d\phi^{2} = d\phi^{2},
\end{equation}
which is precisely the quantum Fisher metric (equivalently, the 
Fubini--Study metric up to a constant factor) for equatorial pure states
\begin{equation}
|\psi(\phi)\rangle
=
\frac{1}{\sqrt{2}}\!\left(|0\rangle + e^{i\phi}|1\rangle\right).
\end{equation}
~\cite{Helstrom1976,Holevo1982,BraunsteinCaves1994,Wootters1981,Fubini1904,Study1905}

Therefore, the DFT phase-geometry not only reproduces the 
cosine–squared beamsplitter response but also matches the 
\emph{information geometry} of quantum phase estimation.  
The beamsplitter is thus sensitive to the same local statistical 
structure that underlies the quantum Cramér--Rao bound.
~\cite{Helstrom1976,Holevo1982,BraunsteinCaves1994}

\subsection{Full Fisher Geometry of the Two-Mode Phase Manifold}
\label{appendix:fisher-2d}

The equatorial Fisher metric derived in Appendix~\ref{appendix:fisher-equator}
extends naturally to the full two-parameter phase manifold when the
DFT subsystem admits two independent geometric degrees of freedom:
a mixing angle $\theta$ and a relative T-frame phase $\phi$.  
This section shows that, once the beamsplitter and phase-shift operations are
interpreted as geometric probes of this $(\theta,\phi)$ manifold, the resulting
Fisher information metric coincides---up to an overall scale---with the
Fubini--Study metric of a qubit,
\begin{equation}
ds^{2}
= d\theta^{2} + \sin^{2}\theta\, d\phi^{2}.
\label{eq:bloch-metric}
\end{equation}
~\cite{Wootters1981,Fubini1904,Study1905,BengtssonZyczkowski2006,AmariNagaoka2000}
Thus the DFT two-mode phase system realizes the full information geometry of
quantum pure states without presupposing a Hilbert-space structure.
~\cite{BengtssonZyczkowski2006}

\subsubsection*{DFT State Manifold and Measurement Families}

A two-mode T-frame subsystem is parameterized by
\begin{equation}
\Psi(\theta,\phi)
\;\hat{=}\;
\begin{pmatrix}
\cos(\tfrac{\theta}{2}) \\
e^{i\phi}\sin(\tfrac{\theta}{2})
\end{pmatrix},
\qquad
0\le\theta\le\pi,\; 0\le\phi<2\pi,
\label{eq:dfp-state}
\end{equation}
which compactly encodes the fraction of scalar progression assigned to
each mode and their relative winding.  
No superposition principle is assumed; Eq.~\eqref{eq:dfp-state} is a
bookkeeping device for the underlying T-frame geometry.

Three physically natural measurement families probe this manifold:

\begin{itemize}
\item \textbf{(z) Direct mode detection:}
\begin{equation}
p^{(z)}_{\pm}(\theta,\phi)
=
\tfrac{1}{2}\!\left[1 \pm \cos\theta\right].
\end{equation}

\item \textbf{(x) Balanced beamsplitter:}
\begin{equation}
p^{(x)}_{\pm}(\theta,\phi)
=
\tfrac{1}{2}\!\left[1 \pm \sin\theta\cos\phi\right].
\end{equation}

\item \textbf{(y) $\pi/2$ phase shift + beamsplitter:}
\begin{equation}
p^{(y)}_{\pm}(\theta,\phi)
=
\tfrac{1}{2}\!\left[1 \pm \sin\theta\sin\phi\right].
\end{equation}
\end{itemize}

Together, these three settings are informationally complete: they resolve
both T-frame phase coordinates using only operations already present in
Appendix~\ref{appendix:beamsplitter}.

\subsubsection*{Fisher Metric from DFT Detection Statistics}

For a measurement family $m\in\{x,y,z\}$ with outcomes $k=\pm$ and
probabilities $p^{(m)}_{k}(\theta,\phi)$,
the Fisher information matrix is
\begin{equation}
g_{ij}
=
\sum_{m}\, w_{m}
\sum_{k=\pm}
\frac{1}{p^{(m)}_{k}}
\left(\partial_{i} p^{(m)}_{k}\right)
\left(\partial_{j} p^{(m)}_{k}\right),
\qquad i,j\in\{\theta,\phi\},
\label{eq:fisher-2d-def}
\end{equation}
with symmetric weights $w_{m}=1/3$.
Each measurement can be written in the form
\begin{equation}
p^{(m)}_{\pm} = \tfrac{1}{2}\!\left(1 \pm r_{m}\right),
\end{equation}
where
\begin{equation}
r_{x} = \sin\theta\cos\phi,\quad
r_{y} = \sin\theta\sin\phi,\quad
r_{z} = \cos\theta.
\end{equation}
A standard identity for binary families gives
\begin{equation}
\sum_{k=\pm}\!
\frac{(\partial_{i} p_{k})(\partial_{j} p_{k})}{p_{k}}
=
\frac{(\partial_{i} r_{m})(\partial_{j} r_{m})}{1-r_{m}^{2}},
\label{eq:binary-fisher-identity}
\end{equation}
~\cite{CoverThomas2006,AmariNagaoka2000}
reducing the computation to derivatives of $r_{m}$:
\begin{align}
\partial_{\theta} r_{x} &= \cos\theta\cos\phi,
&
\partial_{\phi} r_{x} &= -\sin\theta\sin\phi, \\[4pt]
\partial_{\theta} r_{y} &= \cos\theta\sin\phi,
&
\partial_{\phi} r_{y} &= \sin\theta\cos\phi, \\[4pt]
\partial_{\theta} r_{z} &= -\sin\theta,
&
\partial_{\phi} r_{z} &= 0.
\end{align}

\subsubsection*{Analytic Evaluation and Bloch-Sphere Geometry}

Substituting these into Eq.~\eqref{eq:fisher-2d-def} using
Eq.~\eqref{eq:binary-fisher-identity}, one finds:
\begin{align}
g_{\theta\theta}
&= 1, \\[4pt]
g_{\theta\phi}
&= 0, \\[4pt]
g_{\phi\phi}
&= \sin^{2}\theta.
\end{align}

Thus the DFT statistical manifold carries the metric
\begin{equation}
ds^{2}
=
d\theta^{2} + \sin^{2}\theta\, d\phi^{2},
\label{eq:fisher-dft-final}
\end{equation}
which is identical (up to a scale factor determined by the choice of
weights $w_{m}$) to the Fubini--Study metric on the Bloch sphere.  
Therefore the two-mode DFT phase subsystem is not merely analogous to a
qubit: it is \emph{geometrically isomorphic} to the quantum pure-state
manifold when probed by the same phase-sensitive operations that appear
in the beamsplitter analysis.

\subsubsection*{Implications}

\begin{itemize}
\item The beamsplitter and phase-shift operations of DFT reconstruct the
full information geometry of a qubit without invoking wavefunctions or
superposition axioms.

\item The Fisher metric determines optimal estimation limits; thus the 
Cramér--Rao bounds and phase sensitivity of this DFT subsystem match
their quantum counterparts.

\item The emergence of Bloch-sphere geometry from T-frame phase dynamics
supports the interpretive claim that quantum structure originates from
DFT’s compact phase manifold rather than from postulated Hilbert spaces.
\end{itemize}

This completes the bridge between discrete T-frame phase geometry and
the continuum information geometry characteristic of quantum mechanics.

\subsection{Generalization to Unbalanced Beamsplitters}
\label{subsec:bs-generalization}

For an $R{:}T$ beamsplitter ($R+T = 1$), coincidence probabilities need
not vanish at $\Delta\theta = \pi$.  The appropriate generalization of
Eq.~\eqref{eq:harmonic-form} is
\begin{equation}
P_{\mathrm{same}}^{(R,T)}(\Delta\theta)
=
A(R,T) + B(R,T)\,\cos(\Delta\theta).
\end{equation}
Imposing the normalized symmetry and limiting conditions yields
\begin{equation}
P_{\mathrm{same}}^{(R,T)}(\Delta\theta)
=
R^{2} + T^{2} + 2RT \cos(\Delta\theta),
\end{equation}
~\cite{MandelWolf1995,GerryKnight2005}
and for $R=T=\tfrac{1}{2}$ we recover Eq.~\eqref{eq:cos2-appendix}.

This prediction could be tested experimentally using variable
beamsplitters, potentially offering a quantitative probe of the T-frame
phase geometry.

\subsection{Summary of Conceptual Role}

\begin{itemize}
\item The HOM interference rule arises uniquely from symmetry and
normalization constraints applied to T-frame phase relations.
\item No wavefunction, superposition principle, or Hilbert tensor
product is invoked.
\item The Fisher metric interpretation suggests that beamsplitters
probe geometric distinguishability in the T-frame.
\item The unbalanced ($R{:}T$) generalization makes testable predictions.
\end{itemize}

Taken together, these observations clarify how a beamsplitter couples to
relative phase geometry in DFT, enabling the emergent two-particle
interference phenomena central to this paper.

\bibliographystyle{unsrt}   % or journal-required style
\bibliography{references}

\end{document}